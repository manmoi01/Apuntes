


% Header, overrides base

    % Make sure that the sphinx doc style knows who it inherits from.
    \def\sphinxdocclass{article}

    % Declare the document class
    \documentclass[letterpaper,10pt,english]{/usr/lib/python3.3/site-packages/sphinx/texinputs/sphinxhowto}

    % Imports
    \usepackage[utf8]{inputenc}
    \DeclareUnicodeCharacter{00A0}{\\nobreakspace}
    \usepackage[T1]{fontenc}
    \usepackage{babel}
    \usepackage{times}
    \usepackage{import}
    \usepackage[Bjarne]{/usr/lib/python3.3/site-packages/sphinx/texinputs/fncychap}
    \usepackage{longtable}
    \usepackage{/usr/lib/python3.3/site-packages/sphinx/texinputs/sphinx}
    \usepackage{multirow}

    \usepackage{amsmath}
    \usepackage{amssymb}
    \usepackage{ucs}
    \usepackage{enumerate}

    % Used to make the Input/Output rules follow around the contents.
    \usepackage{needspace}

    % Pygments requirements
    \usepackage{fancyvrb}
    \usepackage{color}
    % ansi colors additions
    \definecolor{darkgreen}{rgb}{.12,.54,.11}
    \definecolor{lightgray}{gray}{.95}
    \definecolor{brown}{rgb}{0.54,0.27,0.07}
    \definecolor{purple}{rgb}{0.5,0.0,0.5}
    \definecolor{darkgray}{gray}{0.25}
    \definecolor{lightred}{rgb}{1.0,0.39,0.28}
    \definecolor{lightgreen}{rgb}{0.48,0.99,0.0}
    \definecolor{lightblue}{rgb}{0.53,0.81,0.92}
    \definecolor{lightpurple}{rgb}{0.87,0.63,0.87}
    \definecolor{lightcyan}{rgb}{0.5,1.0,0.83}

    % Needed to box output/input
    \usepackage{tikz}
        \usetikzlibrary{calc,arrows,shadows}
    \usepackage[framemethod=tikz]{mdframed}

    \usepackage{alltt}

    % Used to load and display graphics
    \usepackage{graphicx}
    \graphicspath{ {figs/} }
    \usepackage[Export]{adjustbox} % To resize

    % used so that images for notebooks which have spaces in the name can still be included
    \usepackage{grffile}


    % For formatting output while also word wrapping.
    \usepackage{listings}
    \lstset{breaklines=true}
    \lstset{basicstyle=\small\ttfamily}
    \def\smaller{\fontsize{9.5pt}{9.5pt}\selectfont}

    %Pygments definitions
    
\makeatletter
\def\PY@reset{\let\PY@it=\relax \let\PY@bf=\relax%
    \let\PY@ul=\relax \let\PY@tc=\relax%
    \let\PY@bc=\relax \let\PY@ff=\relax}
\def\PY@tok#1{\csname PY@tok@#1\endcsname}
\def\PY@toks#1+{\ifx\relax#1\empty\else%
    \PY@tok{#1}\expandafter\PY@toks\fi}
\def\PY@do#1{\PY@bc{\PY@tc{\PY@ul{%
    \PY@it{\PY@bf{\PY@ff{#1}}}}}}}
\def\PY#1#2{\PY@reset\PY@toks#1+\relax+\PY@do{#2}}

\expandafter\def\csname PY@tok@go\endcsname{\def\PY@tc##1{\textcolor[rgb]{0.53,0.53,0.53}{##1}}}
\expandafter\def\csname PY@tok@c\endcsname{\let\PY@it=\textit\def\PY@tc##1{\textcolor[rgb]{0.25,0.50,0.50}{##1}}}
\expandafter\def\csname PY@tok@gh\endcsname{\let\PY@bf=\textbf\def\PY@tc##1{\textcolor[rgb]{0.00,0.00,0.50}{##1}}}
\expandafter\def\csname PY@tok@gi\endcsname{\def\PY@tc##1{\textcolor[rgb]{0.00,0.63,0.00}{##1}}}
\expandafter\def\csname PY@tok@gd\endcsname{\def\PY@tc##1{\textcolor[rgb]{0.63,0.00,0.00}{##1}}}
\expandafter\def\csname PY@tok@ge\endcsname{\let\PY@it=\textit}
\expandafter\def\csname PY@tok@m\endcsname{\def\PY@tc##1{\textcolor[rgb]{0.40,0.40,0.40}{##1}}}
\expandafter\def\csname PY@tok@cm\endcsname{\let\PY@it=\textit\def\PY@tc##1{\textcolor[rgb]{0.25,0.50,0.50}{##1}}}
\expandafter\def\csname PY@tok@cs\endcsname{\let\PY@it=\textit\def\PY@tc##1{\textcolor[rgb]{0.25,0.50,0.50}{##1}}}
\expandafter\def\csname PY@tok@cp\endcsname{\def\PY@tc##1{\textcolor[rgb]{0.74,0.48,0.00}{##1}}}
\expandafter\def\csname PY@tok@s\endcsname{\def\PY@tc##1{\textcolor[rgb]{0.73,0.13,0.13}{##1}}}
\expandafter\def\csname PY@tok@w\endcsname{\def\PY@tc##1{\textcolor[rgb]{0.73,0.73,0.73}{##1}}}
\expandafter\def\csname PY@tok@gt\endcsname{\def\PY@tc##1{\textcolor[rgb]{0.00,0.27,0.87}{##1}}}
\expandafter\def\csname PY@tok@gu\endcsname{\let\PY@bf=\textbf\def\PY@tc##1{\textcolor[rgb]{0.50,0.00,0.50}{##1}}}
\expandafter\def\csname PY@tok@gr\endcsname{\def\PY@tc##1{\textcolor[rgb]{1.00,0.00,0.00}{##1}}}
\expandafter\def\csname PY@tok@gs\endcsname{\let\PY@bf=\textbf}
\expandafter\def\csname PY@tok@gp\endcsname{\let\PY@bf=\textbf\def\PY@tc##1{\textcolor[rgb]{0.00,0.00,0.50}{##1}}}
\expandafter\def\csname PY@tok@ne\endcsname{\let\PY@bf=\textbf\def\PY@tc##1{\textcolor[rgb]{0.82,0.25,0.23}{##1}}}
\expandafter\def\csname PY@tok@nd\endcsname{\def\PY@tc##1{\textcolor[rgb]{0.67,0.13,1.00}{##1}}}
\expandafter\def\csname PY@tok@nf\endcsname{\def\PY@tc##1{\textcolor[rgb]{0.00,0.00,1.00}{##1}}}
\expandafter\def\csname PY@tok@na\endcsname{\def\PY@tc##1{\textcolor[rgb]{0.49,0.56,0.16}{##1}}}
\expandafter\def\csname PY@tok@nc\endcsname{\let\PY@bf=\textbf\def\PY@tc##1{\textcolor[rgb]{0.00,0.00,1.00}{##1}}}
\expandafter\def\csname PY@tok@nb\endcsname{\def\PY@tc##1{\textcolor[rgb]{0.00,0.50,0.00}{##1}}}
\expandafter\def\csname PY@tok@nl\endcsname{\def\PY@tc##1{\textcolor[rgb]{0.63,0.63,0.00}{##1}}}
\expandafter\def\csname PY@tok@no\endcsname{\def\PY@tc##1{\textcolor[rgb]{0.53,0.00,0.00}{##1}}}
\expandafter\def\csname PY@tok@nn\endcsname{\let\PY@bf=\textbf\def\PY@tc##1{\textcolor[rgb]{0.00,0.00,1.00}{##1}}}
\expandafter\def\csname PY@tok@ni\endcsname{\let\PY@bf=\textbf\def\PY@tc##1{\textcolor[rgb]{0.60,0.60,0.60}{##1}}}
\expandafter\def\csname PY@tok@sb\endcsname{\def\PY@tc##1{\textcolor[rgb]{0.73,0.13,0.13}{##1}}}
\expandafter\def\csname PY@tok@sc\endcsname{\def\PY@tc##1{\textcolor[rgb]{0.73,0.13,0.13}{##1}}}
\expandafter\def\csname PY@tok@nv\endcsname{\def\PY@tc##1{\textcolor[rgb]{0.10,0.09,0.49}{##1}}}
\expandafter\def\csname PY@tok@sd\endcsname{\let\PY@it=\textit\def\PY@tc##1{\textcolor[rgb]{0.73,0.13,0.13}{##1}}}
\expandafter\def\csname PY@tok@se\endcsname{\let\PY@bf=\textbf\def\PY@tc##1{\textcolor[rgb]{0.73,0.40,0.13}{##1}}}
\expandafter\def\csname PY@tok@mf\endcsname{\def\PY@tc##1{\textcolor[rgb]{0.40,0.40,0.40}{##1}}}
\expandafter\def\csname PY@tok@sh\endcsname{\def\PY@tc##1{\textcolor[rgb]{0.73,0.13,0.13}{##1}}}
\expandafter\def\csname PY@tok@si\endcsname{\let\PY@bf=\textbf\def\PY@tc##1{\textcolor[rgb]{0.73,0.40,0.53}{##1}}}
\expandafter\def\csname PY@tok@sr\endcsname{\def\PY@tc##1{\textcolor[rgb]{0.73,0.40,0.53}{##1}}}
\expandafter\def\csname PY@tok@k\endcsname{\let\PY@bf=\textbf\def\PY@tc##1{\textcolor[rgb]{0.00,0.50,0.00}{##1}}}
\expandafter\def\csname PY@tok@err\endcsname{\def\PY@bc##1{\setlength{\fboxsep}{0pt}\fcolorbox[rgb]{1.00,0.00,0.00}{1,1,1}{\strut ##1}}}
\expandafter\def\csname PY@tok@bp\endcsname{\def\PY@tc##1{\textcolor[rgb]{0.00,0.50,0.00}{##1}}}
\expandafter\def\csname PY@tok@c1\endcsname{\let\PY@it=\textit\def\PY@tc##1{\textcolor[rgb]{0.25,0.50,0.50}{##1}}}
\expandafter\def\csname PY@tok@o\endcsname{\def\PY@tc##1{\textcolor[rgb]{0.40,0.40,0.40}{##1}}}
\expandafter\def\csname PY@tok@sx\endcsname{\def\PY@tc##1{\textcolor[rgb]{0.00,0.50,0.00}{##1}}}
\expandafter\def\csname PY@tok@s2\endcsname{\def\PY@tc##1{\textcolor[rgb]{0.73,0.13,0.13}{##1}}}
\expandafter\def\csname PY@tok@s1\endcsname{\def\PY@tc##1{\textcolor[rgb]{0.73,0.13,0.13}{##1}}}
\expandafter\def\csname PY@tok@mh\endcsname{\def\PY@tc##1{\textcolor[rgb]{0.40,0.40,0.40}{##1}}}
\expandafter\def\csname PY@tok@mi\endcsname{\def\PY@tc##1{\textcolor[rgb]{0.40,0.40,0.40}{##1}}}
\expandafter\def\csname PY@tok@mo\endcsname{\def\PY@tc##1{\textcolor[rgb]{0.40,0.40,0.40}{##1}}}
\expandafter\def\csname PY@tok@kr\endcsname{\let\PY@bf=\textbf\def\PY@tc##1{\textcolor[rgb]{0.00,0.50,0.00}{##1}}}
\expandafter\def\csname PY@tok@kp\endcsname{\def\PY@tc##1{\textcolor[rgb]{0.00,0.50,0.00}{##1}}}
\expandafter\def\csname PY@tok@il\endcsname{\def\PY@tc##1{\textcolor[rgb]{0.40,0.40,0.40}{##1}}}
\expandafter\def\csname PY@tok@kt\endcsname{\def\PY@tc##1{\textcolor[rgb]{0.69,0.00,0.25}{##1}}}
\expandafter\def\csname PY@tok@ow\endcsname{\let\PY@bf=\textbf\def\PY@tc##1{\textcolor[rgb]{0.67,0.13,1.00}{##1}}}
\expandafter\def\csname PY@tok@vi\endcsname{\def\PY@tc##1{\textcolor[rgb]{0.10,0.09,0.49}{##1}}}
\expandafter\def\csname PY@tok@kd\endcsname{\let\PY@bf=\textbf\def\PY@tc##1{\textcolor[rgb]{0.00,0.50,0.00}{##1}}}
\expandafter\def\csname PY@tok@kn\endcsname{\let\PY@bf=\textbf\def\PY@tc##1{\textcolor[rgb]{0.00,0.50,0.00}{##1}}}
\expandafter\def\csname PY@tok@ss\endcsname{\def\PY@tc##1{\textcolor[rgb]{0.10,0.09,0.49}{##1}}}
\expandafter\def\csname PY@tok@kc\endcsname{\let\PY@bf=\textbf\def\PY@tc##1{\textcolor[rgb]{0.00,0.50,0.00}{##1}}}
\expandafter\def\csname PY@tok@vg\endcsname{\def\PY@tc##1{\textcolor[rgb]{0.10,0.09,0.49}{##1}}}
\expandafter\def\csname PY@tok@nt\endcsname{\let\PY@bf=\textbf\def\PY@tc##1{\textcolor[rgb]{0.00,0.50,0.00}{##1}}}
\expandafter\def\csname PY@tok@vc\endcsname{\def\PY@tc##1{\textcolor[rgb]{0.10,0.09,0.49}{##1}}}

\def\PYZbs{\char`\\}
\def\PYZus{\char`\_}
\def\PYZob{\char`\{}
\def\PYZcb{\char`\}}
\def\PYZca{\char`\^}
\def\PYZam{\char`\&}
\def\PYZlt{\char`\<}
\def\PYZgt{\char`\>}
\def\PYZsh{\char`\#}
\def\PYZpc{\char`\%}
\def\PYZdl{\char`\$}
\def\PYZhy{\char`\-}
\def\PYZsq{\char`\'}
\def\PYZdq{\char`\"}
\def\PYZti{\char`\~}
% for compatibility with earlier versions
\def\PYZat{@}
\def\PYZlb{[}
\def\PYZrb{]}
\makeatother


    %Set pygments styles if needed...
    
        \definecolor{nbframe-border}{rgb}{0.867,0.867,0.867}
        \definecolor{nbframe-bg}{rgb}{0.969,0.969,0.969}
        \definecolor{nbframe-in-prompt}{rgb}{0.0,0.0,0.502}
        \definecolor{nbframe-out-prompt}{rgb}{0.545,0.0,0.0}

        \newenvironment{ColorVerbatim}
        {\begin{mdframed}[%
            roundcorner=1.0pt, %
            backgroundcolor=nbframe-bg, %
            userdefinedwidth=1\linewidth, %
            leftmargin=0.1\linewidth, %
            innerleftmargin=0pt, %
            innerrightmargin=0pt, %
            linecolor=nbframe-border, %
            linewidth=1pt, %
            usetwoside=false, %
            everyline=true, %
            innerlinewidth=3pt, %
            innerlinecolor=nbframe-bg, %
            middlelinewidth=1pt, %
            middlelinecolor=nbframe-bg, %
            outerlinewidth=0.5pt, %
            outerlinecolor=nbframe-border, %
            needspace=0pt
        ]}
        {\end{mdframed}}
        
        \newenvironment{InvisibleVerbatim}
        {\begin{mdframed}[leftmargin=0.1\linewidth,innerleftmargin=3pt,innerrightmargin=3pt, userdefinedwidth=1\linewidth, linewidth=0pt, linecolor=white, usetwoside=false]}
        {\end{mdframed}}

        \renewenvironment{Verbatim}[1][\unskip]
        {\begin{alltt}\smaller}
        {\end{alltt}}
    

    % Help prevent overflowing lines due to urls and other hard-to-break 
    % entities.  This doesn't catch everything...
    \sloppy

    % Document level variables
    \title{Circuitos usando Diodos}
    \date{August 31, 2013}
    \release{}
    \author{Unknown Author}
    \renewcommand{\releasename}{}

    % TODO: Add option for the user to specify a logo for his/her export.
    \newcommand{\sphinxlogo}{}

    % Make the index page of the document.
    \makeindex

    % Import sphinx document type specifics.
     


% Body

    % Start of the document
    \begin{document}

        
            \maketitle
        

        


        
        \section{Curvas Caracteristicas}

Con un circuito simple se puede utilizar el circuito que se muestra
acontinuación:

\begin{figure}[htbp]
\centering
\includegraphics{/files/images/curvadiodocir.png}
\caption{Circuito Básico}
\end{figure}

dado que la fuente de corriente puede tener cualquier valor de voltaje
(teoricamente), si controlamos la corriente $I_{D}$ entonces obtendremos
una variación en el voltaje del diodo $V_{D}$, la cual debe de ser
similar al expresado por la expresión:

\begin{equation}\label{eq:diodo}
I_{D}=I_{s}\left( e^{\frac{V_{D}}{nV_{T}}}-1\right)
\end{equation}

Utilizando el program LTspice para obtener el comportamiento de
diferentes diodos se obtiene la siguiente gráfica. (Los archivos son:
\href{http://cacie.ens.uabc.mx/~mmiranda/cursos/notas/Electronica/diodos_curva.asc}{diodos\_curvas.asc}
el archivo de LTspice y el de
\href{http://cacie.ens.uabc.mx/~mmiranda/cursos/notas/Electronica/curvasdiodos.txt}{curvasdiodos.txt}
contiene los datos que se usan en el ejemplo.

    % Make sure that atleast 4 lines are below the HR
    \needspace{4\baselineskip}

    
        \vspace{6pt}
        \makebox[0.1\linewidth]{\smaller\hfill\tt\color{nbframe-in-prompt}In\hspace{4pt}{[}11{]}:\hspace{4pt}}\\*
        \vspace{-2.65\baselineskip}
        \begin{ColorVerbatim}
            \vspace{-0.7\baselineskip}
            \begin{Verbatim}[commandchars=\\\{\}]
\PY{k+kn}{import} \PY{n+nn}{numpy} \PY{k+kn}{as} \PY{n+nn}{np}
\PY{k+kn}{import} \PY{n+nn}{pylab}
\PY{n}{varia}\PY{o}{=}\PY{n}{np}\PY{o}{.}\PY{n}{loadtxt}\PY{p}{(}\PY{l+s}{\PYZsq{}}\PY{l+s}{curvasdiodos.txt}\PY{l+s}{\PYZsq{}}\PY{p}{,} \PY{n}{skiprows}\PY{o}{=}\PY{l+m+mi}{1}\PY{p}{)}
\PY{c}{\PYZsh{}matplotlib.rcParams.update(\PYZob{}\PYZsq{}font.size\PYZsq{}: 20,\PYZsq{}text.usetex\PYZsq{}: True\PYZcb{})}
\PY{n}{matplotlib}\PY{o}{.}\PY{n}{rcParams}\PY{o}{.}\PY{n}{update}\PY{p}{(}\PY{p}{\PYZob{}}\PY{l+s}{\PYZsq{}}\PY{l+s}{text.usetex}\PY{l+s}{\PYZsq{}}\PY{p}{:} \PY{n+nb+bp}{True}\PY{p}{\PYZcb{}}\PY{p}{)}
\PY{c}{\PYZsh{}figure(figsize=(14,10), dpi=150)}
\PY{n}{xlabel}\PY{p}{(}\PY{l+s}{\PYZsq{}}\PY{l+s}{\PYZdl{}V\PYZus{}\PYZob{}d\PYZcb{}\PYZdl{}}\PY{l+s}{\PYZsq{}}\PY{p}{)}
\PY{n}{ylabel}\PY{p}{(}\PY{l+s}{\PYZsq{}}\PY{l+s}{\PYZdl{}I\PYZus{}\PYZob{}d\PYZcb{}\PYZdl{}}\PY{l+s}{\PYZsq{}}\PY{p}{)}
\PY{n}{xlim}\PY{p}{(}\PY{l+m+mi}{0}\PY{p}{,}\PY{l+m+mi}{4}\PY{p}{)}
\PY{n}{ylim}\PY{p}{(}\PY{l+m+mi}{0}\PY{p}{,}\PY{l+m+mf}{0.1}\PY{p}{)}
\PY{n}{plot}\PY{p}{(}\PY{n}{varia}\PY{p}{[}\PY{p}{:}\PY{p}{,}\PY{l+m+mi}{1}\PY{p}{]}\PY{p}{,}\PY{n}{varia}\PY{p}{[}\PY{p}{:}\PY{p}{,}\PY{l+m+mi}{0}\PY{p}{]}\PY{p}{,}\PY{n}{linewidth}\PY{o}{=}\PY{l+m+mi}{3}\PY{p}{,}\PY{n}{label}\PY{o}{=}\PY{l+s}{\PYZsq{}}\PY{l+s}{Ideal}\PY{l+s}{\PYZsq{}}\PY{p}{)}
\PY{n}{plot}\PY{p}{(}\PY{n}{varia}\PY{p}{[}\PY{p}{:}\PY{p}{,}\PY{l+m+mi}{2}\PY{p}{]}\PY{p}{,}\PY{n}{varia}\PY{p}{[}\PY{p}{:}\PY{p}{,}\PY{l+m+mi}{0}\PY{p}{]}\PY{p}{,}\PY{n}{linewidth}\PY{o}{=}\PY{l+m+mi}{3}\PY{p}{,}\PY{n}{label}\PY{o}{=}\PY{l+s}{\PYZsq{}}\PY{l+s}{LED}\PY{l+s}{\PYZsq{}}\PY{p}{)}
\PY{n}{plot}\PY{p}{(}\PY{n}{varia}\PY{p}{[}\PY{p}{:}\PY{p}{,}\PY{l+m+mi}{3}\PY{p}{]}\PY{p}{,}\PY{n}{varia}\PY{p}{[}\PY{p}{:}\PY{p}{,}\PY{l+m+mi}{0}\PY{p}{]}\PY{p}{,}\PY{n}{linewidth}\PY{o}{=}\PY{l+m+mi}{3}\PY{p}{,}\PY{n}{label}\PY{o}{=}\PY{l+s}{\PYZsq{}}\PY{l+s}{1N4148}\PY{l+s}{\PYZsq{}}\PY{p}{)}
\PY{n}{plot}\PY{p}{(}\PY{n}{varia}\PY{p}{[}\PY{p}{:}\PY{p}{,}\PY{l+m+mi}{4}\PY{p}{]}\PY{p}{,}\PY{n}{varia}\PY{p}{[}\PY{p}{:}\PY{p}{,}\PY{l+m+mi}{0}\PY{p}{]}\PY{p}{,}\PY{n}{linewidth}\PY{o}{=}\PY{l+m+mi}{3}\PY{p}{,}\PY{n}{label}\PY{o}{=}\PY{l+s}{\PYZsq{}}\PY{l+s}{Schotky}\PY{l+s}{\PYZsq{}}\PY{p}{)}
\PY{n}{legend}\PY{p}{(}\PY{n}{loc}\PY{o}{=}\PY{l+m+mi}{9}\PY{p}{)}
\PY{n}{grid}\PY{p}{(}\PY{p}{)}
\end{Verbatim}

            
                \vspace{-0.2\baselineskip}
            
        \end{ColorVerbatim}
    

    

        % If the first block is an image, minipage the image.  Else
        % request a certain amount of space for the input text.
        \needspace{4\baselineskip}
        
        

            % Add document contents.
            
                \begin{InvisibleVerbatim}
                \vspace{-0.5\baselineskip}
    \begin{center}
    \includegraphics[max size={\textwidth}{\textheight}]{Circuitos usando Diodos_files/Circuitos usando Diodos_1_0.png}
    \par
    \end{center}
    
            \end{InvisibleVerbatim}
            
        
    
\section{Circuito Rectificador de media onda}

El circuito más simple con un diodo es el conocido como rectificador de
media onda el cual consiste de una fuente de voltaje $V_{i}$, un
resistencia $R$ y un diodo $D$ en seria, como se muestra en la siguiente
figura

\begin{figure}[htbp]
\centering
\includegraphics{/files/images/diodosimple.png}
\caption{Circuito con un diodo}
\end{figure}

Utilizando la ley de Kircchof de voltajes y la ley de Ohm se tiene

\begin{equation}\label{eq:uno}
V_{i}=I_{D}R+V_{D}
\end{equation}

dado que los elementos estan en serie se tiene que corriente es la
misma, por lo tanto es posible encontrar la corriente total con la
expresión para la corriente del diodo

\begin{equation}\label{eq:id}
I_{D}=I_{s}\left( e^{\frac{V_{D}}{nV_{T}}}-1\right)
\end{equation}

sustituyendo \eqref{eq:id} en \eqref{eq:uno}

\begin{equation}\label{eq:dos}
V_{i}=I_{s}\left( e^{\frac{V_{D}}{nV_{T}}}-1\right)R+V_{D}
\end{equation}

La expresión \eqref{eq:dos} es una ecuación trascendente (Una ecuación
trascendente es una igualdad entre dos expresiones matemáticas en las
que aparecen una o más incógnitas relacionadas mediante operaciones
matemáticas, que no son únicamente algebraicas, y cuya solución no puede
obtenerse empleando solo las herramientas propias del álgebra.) lo cual
no permite encontrar una solución analitica.

Para resolver una ecuación trascendente se utilizan dos métodos que son:
por estimación y gráfico.

\subsection{Estimación}

Considerese una expresión del tipo

\begin{equation}\label{eq:idea1}
y=f(x,p)
\end{equation}

donde $y$ es una salida conocida (Variable dependiente), $x$ es una
entrada (Variable independiente) y $p$ son un conjunto de parámetros. Si
$f(x,p)$ es una ecuación que no tiene solución analitica, entonces no se
puede calcular su solución exacta. Sin embargo, es posible estimar una
solución aproximada $\hat{y}$ con el método de estimación. En este
método se asignan valores a la incognita hasta que el error ($E_{y}$)
entre la solución exacta $y$ y la solución estimada $\hat{y}$ es menor a
un cierto valor establecido. Esto se puede expresar como:

\[
E_{y}=\vert y-\hat{y}\vert \leq k
\]

Por lo que este metodo requiere que los coeficientes de la expresión a
evular sea conocidos.

\subsubsection{Ejemplo}

en el caso de la expresión \eqref{eq:dos} consideramos que $R=2k\Omega$,
$V_{i}=5$, $n=2$, y $V_{T}=26mV$. En la siguiente gráfica se muestra el
valor estimado de la expresión \eqref{eq:dos} al sustituir valores de
los parámetros y considerar $0\leq V_{D} \leq 1$

    % Make sure that atleast 4 lines are below the HR
    \needspace{4\baselineskip}

    
        \vspace{6pt}
        \makebox[0.1\linewidth]{\smaller\hfill\tt\color{nbframe-in-prompt}In\hspace{4pt}{[}2{]}:\hspace{4pt}}\\*
        \vspace{-2.65\baselineskip}
        \begin{ColorVerbatim}
            \vspace{-0.7\baselineskip}
            \begin{Verbatim}[commandchars=\\\{\}]
\PY{n}{R}\PY{o}{=}\PY{n}{np}\PY{o}{.}\PY{n}{array}\PY{p}{(}\PY{p}{[}\PY{l+m+mf}{2e3}\PY{p}{]}\PY{p}{)}
\PY{n}{n}\PY{o}{=}\PY{n}{np}\PY{o}{.}\PY{n}{array}\PY{p}{(}\PY{p}{[}\PY{l+m+mi}{2}\PY{p}{]}\PY{p}{)}
\PY{n}{Vt}\PY{o}{=}\PY{n}{np}\PY{o}{.}\PY{n}{array}\PY{p}{(}\PY{p}{[}\PY{l+m+mf}{26e\PYZhy{}3}\PY{p}{]}\PY{p}{)}
\PY{n}{Is}\PY{o}{=}\PY{n}{np}\PY{o}{.}\PY{n}{array}\PY{p}{(}\PY{p}{[}\PY{o}{.}\PY{l+m+mf}{05e\PYZhy{}6}\PY{p}{]}\PY{p}{)}
\PY{n}{Vd}\PY{o}{=}\PY{n}{np}\PY{o}{.}\PY{n}{linspace}\PY{p}{(}\PY{l+m+mi}{0}\PY{p}{,} \PY{l+m+mi}{1}\PY{p}{,} \PY{l+m+mi}{10}\PY{p}{)}
\PY{n}{Vi}\PY{o}{=}\PY{n}{np}\PY{o}{.}\PY{n}{ones}\PY{p}{(}\PY{n+nb}{len}\PY{p}{(}\PY{n}{Vd}\PY{p}{)}\PY{p}{)}\PY{o}{*}\PY{l+m+mi}{5}
\PY{n}{yp}\PY{o}{=}\PY{n}{Is}\PY{o}{*}\PY{p}{(}\PY{n}{np}\PY{o}{.}\PY{n}{exp}\PY{p}{(}\PY{n}{Vd}\PY{o}{/}\PY{p}{(}\PY{n}{n}\PY{o}{*}\PY{n}{Vt}\PY{p}{)}\PY{p}{)}\PY{o}{\PYZhy{}}\PY{l+m+mi}{1}\PY{p}{)}\PY{o}{*}\PY{n}{R}\PY{o}{+}\PY{n}{Vd}
\PY{n}{matplotlib}\PY{o}{.}\PY{n}{rcParams}\PY{o}{.}\PY{n}{update}\PY{p}{(}\PY{p}{\PYZob{}}\PY{l+s}{\PYZsq{}}\PY{l+s}{font.size}\PY{l+s}{\PYZsq{}}\PY{p}{:} \PY{l+m+mi}{18}\PY{p}{\PYZcb{}}\PY{p}{)}
\PY{n}{figure}\PY{p}{(}\PY{n}{figsize}\PY{o}{=}\PY{p}{(}\PY{l+m+mi}{14}\PY{p}{,}\PY{l+m+mi}{10}\PY{p}{)}\PY{p}{,} \PY{n}{dpi}\PY{o}{=}\PY{l+m+mi}{150}\PY{p}{)}
\PY{n}{xlabel}\PY{p}{(}\PY{l+s}{\PYZsq{}}\PY{l+s}{Vd}\PY{l+s}{\PYZsq{}}\PY{p}{)}
\PY{n}{ylabel}\PY{p}{(}\PY{l+s}{\PYZsq{}}\PY{l+s}{Vi}\PY{l+s}{\PYZsq{}}\PY{p}{)}
\PY{n}{plot}\PY{p}{(}\PY{n}{Vd}\PY{p}{,}\PY{n}{yp}\PY{p}{,}\PY{n}{linewidth}\PY{o}{=}\PY{l+m+mi}{3}\PY{p}{)}
\PY{n}{plot}\PY{p}{(}\PY{n}{Vd}\PY{p}{,}\PY{n}{Vi}\PY{p}{,}\PY{n}{linewidth}\PY{o}{=}\PY{l+m+mi}{3}\PY{p}{)}
\PY{n}{grid}\PY{p}{(}\PY{p}{)}
\end{Verbatim}

            
                \vspace{-0.2\baselineskip}
            
        \end{ColorVerbatim}
    

    

        % If the first block is an image, minipage the image.  Else
        % request a certain amount of space for the input text.
        \needspace{4\baselineskip}
        
        

            % Add document contents.
            
                \begin{InvisibleVerbatim}
                \vspace{-0.5\baselineskip}
    \begin{center}
    \includegraphics[max size={\textwidth}{\textheight}]{Circuitos usando Diodos_files/Circuitos usando Diodos_3_0.png}
    \par
    \end{center}
    
            \end{InvisibleVerbatim}
            
        
    
De la gráfica anterior se observa que para valores de $V_{D}$ mayores a
0.8, la salida estimada es mucho mayor que la salida esperada, por lo
que haciendo una nueva iteración $0\leq V_{D} \leq 0.7$, se obtiene la
siguiente gráfica

    % Make sure that atleast 4 lines are below the HR
    \needspace{4\baselineskip}

    
        \vspace{6pt}
        \makebox[0.1\linewidth]{\smaller\hfill\tt\color{nbframe-in-prompt}In\hspace{4pt}{[}3{]}:\hspace{4pt}}\\*
        \vspace{-2.65\baselineskip}
        \begin{ColorVerbatim}
            \vspace{-0.7\baselineskip}
            \begin{Verbatim}[commandchars=\\\{\}]
\PY{n}{Vd}\PY{o}{=}\PY{n}{np}\PY{o}{.}\PY{n}{linspace}\PY{p}{(}\PY{l+m+mi}{0}\PY{p}{,} \PY{o}{.}\PY{l+m+mi}{7}\PY{p}{,} \PY{l+m+mi}{100}\PY{p}{)}
\PY{n}{Vi}\PY{o}{=}\PY{n}{np}\PY{o}{.}\PY{n}{ones}\PY{p}{(}\PY{n+nb}{len}\PY{p}{(}\PY{n}{Vd}\PY{p}{)}\PY{p}{)}\PY{o}{*}\PY{l+m+mi}{5}
\PY{n}{yp}\PY{o}{=}\PY{n}{Is}\PY{o}{*}\PY{p}{(}\PY{n}{np}\PY{o}{.}\PY{n}{exp}\PY{p}{(}\PY{n}{Vd}\PY{o}{/}\PY{p}{(}\PY{n}{n}\PY{o}{*}\PY{n}{Vt}\PY{p}{)}\PY{p}{)}\PY{o}{\PYZhy{}}\PY{l+m+mi}{1}\PY{p}{)}\PY{o}{*}\PY{n}{R}\PY{o}{+}\PY{n}{Vd}
\PY{n}{matplotlib}\PY{o}{.}\PY{n}{rcParams}\PY{o}{.}\PY{n}{update}\PY{p}{(}\PY{p}{\PYZob{}}\PY{l+s}{\PYZsq{}}\PY{l+s}{font.size}\PY{l+s}{\PYZsq{}}\PY{p}{:} \PY{l+m+mi}{18}\PY{p}{\PYZcb{}}\PY{p}{)}
\PY{n}{figure}\PY{p}{(}\PY{n}{figsize}\PY{o}{=}\PY{p}{(}\PY{l+m+mi}{14}\PY{p}{,}\PY{l+m+mi}{10}\PY{p}{)}\PY{p}{,} \PY{n}{dpi}\PY{o}{=}\PY{l+m+mi}{150}\PY{p}{)}
\PY{n}{xlabel}\PY{p}{(}\PY{l+s}{\PYZsq{}}\PY{l+s}{Vd}\PY{l+s}{\PYZsq{}}\PY{p}{)}
\PY{n}{ylabel}\PY{p}{(}\PY{l+s}{\PYZsq{}}\PY{l+s}{Vi}\PY{l+s}{\PYZsq{}}\PY{p}{)}
\PY{n}{plot}\PY{p}{(}\PY{n}{Vd}\PY{p}{,}\PY{n}{yp}\PY{p}{,}\PY{n}{linewidth}\PY{o}{=}\PY{l+m+mi}{3}\PY{p}{)}
\PY{n}{plot}\PY{p}{(}\PY{n}{Vd}\PY{p}{,}\PY{n}{Vi}\PY{p}{,}\PY{n}{linewidth}\PY{o}{=}\PY{l+m+mi}{3}\PY{p}{)}
\PY{n}{grid}\PY{p}{(}\PY{p}{)}
\end{Verbatim}

            
                \vspace{-0.2\baselineskip}
            
        \end{ColorVerbatim}
    

    

        % If the first block is an image, minipage the image.  Else
        % request a certain amount of space for the input text.
        \needspace{4\baselineskip}
        
        

            % Add document contents.
            
                \begin{InvisibleVerbatim}
                \vspace{-0.5\baselineskip}
    \begin{center}
    \includegraphics[max size={\textwidth}{\textheight}]{Circuitos usando Diodos_files/Circuitos usando Diodos_5_0.png}
    \par
    \end{center}
    
            \end{InvisibleVerbatim}
            
        
    
De los resultados mostrados en la gráfica anterior se puede concluir que
el valor estimado se aproxima al valor deseado caundo
$0.5\leq V_{D} \leq 0.6$, para realizar una mejor aproximación se
utiliza el error de estimacion

\begin{equation}
E=V_{i}-\hat{V_{i}}
\end{equation}

donde $\hat{V_{i}}$ es el valor estimado a través de la expresion
\eqref{eq:dos}

    % Make sure that atleast 4 lines are below the HR
    \needspace{4\baselineskip}

    
        \vspace{6pt}
        \makebox[0.1\linewidth]{\smaller\hfill\tt\color{nbframe-in-prompt}In\hspace{4pt}{[}4{]}:\hspace{4pt}}\\*
        \vspace{-2.65\baselineskip}
        \begin{ColorVerbatim}
            \vspace{-0.7\baselineskip}
            \begin{Verbatim}[commandchars=\\\{\}]
\PY{n}{Vd}\PY{o}{=}\PY{n}{np}\PY{o}{.}\PY{n}{linspace}\PY{p}{(}\PY{l+m+mf}{0.5}\PY{p}{,} \PY{o}{.}\PY{l+m+mi}{6}\PY{p}{,} \PY{l+m+mi}{100}\PY{p}{)}
\PY{n}{Vi}\PY{o}{=}\PY{n}{np}\PY{o}{.}\PY{n}{ones}\PY{p}{(}\PY{n+nb}{len}\PY{p}{(}\PY{n}{Vd}\PY{p}{)}\PY{p}{)}\PY{o}{*}\PY{l+m+mi}{5}
\PY{n}{yp}\PY{o}{=}\PY{n}{Is}\PY{o}{*}\PY{p}{(}\PY{n}{np}\PY{o}{.}\PY{n}{exp}\PY{p}{(}\PY{n}{Vd}\PY{o}{/}\PY{p}{(}\PY{n}{n}\PY{o}{*}\PY{n}{Vt}\PY{p}{)}\PY{p}{)}\PY{o}{\PYZhy{}}\PY{l+m+mi}{1}\PY{p}{)}\PY{o}{*}\PY{n}{R}\PY{o}{+}\PY{n}{Vd}
\PY{n}{E}\PY{o}{=}\PY{l+m+mi}{5}\PY{o}{\PYZhy{}}\PY{n}{yp}\PY{p}{;}
\PY{n}{matplotlib}\PY{o}{.}\PY{n}{rcParams}\PY{o}{.}\PY{n}{update}\PY{p}{(}\PY{p}{\PYZob{}}\PY{l+s}{\PYZsq{}}\PY{l+s}{font.size}\PY{l+s}{\PYZsq{}}\PY{p}{:} \PY{l+m+mi}{18}\PY{p}{\PYZcb{}}\PY{p}{)}
\PY{n}{figure}\PY{p}{(}\PY{n}{figsize}\PY{o}{=}\PY{p}{(}\PY{l+m+mi}{14}\PY{p}{,}\PY{l+m+mi}{10}\PY{p}{)}\PY{p}{,} \PY{n}{dpi}\PY{o}{=}\PY{l+m+mi}{150}\PY{p}{)}
\PY{n}{xlim}\PY{p}{(}\PY{l+m+mf}{0.5}\PY{p}{,}\PY{l+m+mf}{0.6}\PY{p}{)}
\PY{n}{xlabel}\PY{p}{(}\PY{l+s}{\PYZsq{}}\PY{l+s}{Vd}\PY{l+s}{\PYZsq{}}\PY{p}{)}
\PY{n}{ylabel}\PY{p}{(}\PY{l+s}{\PYZsq{}}\PY{l+s}{E}\PY{l+s}{\PYZsq{}}\PY{p}{)}
\PY{n}{plot}\PY{p}{(}\PY{n}{Vd}\PY{p}{,}\PY{n}{E}\PY{p}{,}\PY{n}{linewidth}\PY{o}{=}\PY{l+m+mi}{3}\PY{p}{)}
\PY{n}{grid}\PY{p}{(}\PY{p}{)}
\end{Verbatim}

            
                \vspace{-0.2\baselineskip}
            
        \end{ColorVerbatim}
    

    

        % If the first block is an image, minipage the image.  Else
        % request a certain amount of space for the input text.
        \needspace{4\baselineskip}
        
        

            % Add document contents.
            
                \begin{InvisibleVerbatim}
                \vspace{-0.5\baselineskip}
    \begin{center}
    \includegraphics[max size={\textwidth}{\textheight}]{Circuitos usando Diodos_files/Circuitos usando Diodos_7_0.png}
    \par
    \end{center}
    
            \end{InvisibleVerbatim}
            
        
    


    % Make sure that atleast 4 lines are below the HR
    \needspace{4\baselineskip}

    
        \vspace{6pt}
        \makebox[0.1\linewidth]{\smaller\hfill\tt\color{nbframe-in-prompt}In\hspace{4pt}{[}5{]}:\hspace{4pt}}\\*
        \vspace{-2.65\baselineskip}
        \begin{ColorVerbatim}
            \vspace{-0.7\baselineskip}
            \begin{Verbatim}[commandchars=\\\{\}]
\PY{n}{Vd}\PY{o}{=}\PY{n}{np}\PY{o}{.}\PY{n}{linspace}\PY{p}{(}\PY{l+m+mf}{0.55}\PY{p}{,} \PY{o}{.}\PY{l+m+mi}{56}\PY{p}{,} \PY{l+m+mi}{100}\PY{p}{)}
\PY{n}{Vi}\PY{o}{=}\PY{n}{np}\PY{o}{.}\PY{n}{ones}\PY{p}{(}\PY{n+nb}{len}\PY{p}{(}\PY{n}{Vd}\PY{p}{)}\PY{p}{)}\PY{o}{*}\PY{l+m+mi}{5}
\PY{n}{yp}\PY{o}{=}\PY{n}{Is}\PY{o}{*}\PY{p}{(}\PY{n}{np}\PY{o}{.}\PY{n}{exp}\PY{p}{(}\PY{n}{Vd}\PY{o}{/}\PY{p}{(}\PY{n}{n}\PY{o}{*}\PY{n}{Vt}\PY{p}{)}\PY{p}{)}\PY{o}{\PYZhy{}}\PY{l+m+mi}{1}\PY{p}{)}\PY{o}{*}\PY{n}{R}\PY{o}{+}\PY{n}{Vd}
\PY{n}{E}\PY{o}{=}\PY{l+m+mi}{5}\PY{o}{\PYZhy{}}\PY{n}{yp}\PY{p}{;}
\PY{n}{matplotlib}\PY{o}{.}\PY{n}{rcParams}\PY{o}{.}\PY{n}{update}\PY{p}{(}\PY{p}{\PYZob{}}\PY{l+s}{\PYZsq{}}\PY{l+s}{font.size}\PY{l+s}{\PYZsq{}}\PY{p}{:} \PY{l+m+mi}{18}\PY{p}{\PYZcb{}}\PY{p}{)}
\PY{n}{figure}\PY{p}{(}\PY{n}{figsize}\PY{o}{=}\PY{p}{(}\PY{l+m+mi}{14}\PY{p}{,}\PY{l+m+mi}{10}\PY{p}{)}\PY{p}{,} \PY{n}{dpi}\PY{o}{=}\PY{l+m+mi}{150}\PY{p}{)}
\PY{n}{xlabel}\PY{p}{(}\PY{l+s}{\PYZsq{}}\PY{l+s}{Vd}\PY{l+s}{\PYZsq{}}\PY{p}{)}
\PY{n}{ylabel}\PY{p}{(}\PY{l+s}{\PYZsq{}}\PY{l+s}{E}\PY{l+s}{\PYZsq{}}\PY{p}{)}
\PY{n}{plot}\PY{p}{(}\PY{n}{Vd}\PY{p}{,}\PY{n}{E}\PY{p}{,}\PY{n}{linewidth}\PY{o}{=}\PY{l+m+mi}{3}\PY{p}{)}
\PY{n}{grid}\PY{p}{(}\PY{p}{)}
\end{Verbatim}

            
                \vspace{-0.2\baselineskip}
            
        \end{ColorVerbatim}
    

    

        % If the first block is an image, minipage the image.  Else
        % request a certain amount of space for the input text.
        \needspace{4\baselineskip}
        
        

            % Add document contents.
            
                \begin{InvisibleVerbatim}
                \vspace{-0.5\baselineskip}
    \begin{center}
    \includegraphics[max size={\textwidth}{\textheight}]{Circuitos usando Diodos_files/Circuitos usando Diodos_8_0.png}
    \par
    \end{center}
    
            \end{InvisibleVerbatim}
            
        
    


    % Make sure that atleast 4 lines are below the HR
    \needspace{4\baselineskip}

    
        \vspace{6pt}
        \makebox[0.1\linewidth]{\smaller\hfill\tt\color{nbframe-in-prompt}In\hspace{4pt}{[}6{]}:\hspace{4pt}}\\*
        \vspace{-2.65\baselineskip}
        \begin{ColorVerbatim}
            \vspace{-0.7\baselineskip}
            \begin{Verbatim}[commandchars=\\\{\}]
\PY{n}{Vd}\PY{o}{=}\PY{l+m+mf}{0.556495}
\PY{n}{Vi}\PY{o}{=}\PY{l+m+mi}{5}
\PY{n}{yp}\PY{o}{=}\PY{n}{array}\PY{p}{(}\PY{p}{[}\PY{p}{]}\PY{p}{)}
\PY{n}{yp}\PY{o}{=}\PY{n}{Is}\PY{o}{*}\PY{p}{(}\PY{n}{np}\PY{o}{.}\PY{n}{exp}\PY{p}{(}\PY{n}{Vd}\PY{o}{/}\PY{p}{(}\PY{n}{n}\PY{o}{*}\PY{n}{Vt}\PY{p}{)}\PY{p}{)}\PY{o}{\PYZhy{}}\PY{l+m+mi}{1}\PY{p}{)}\PY{o}{*}\PY{n}{R}\PY{o}{+}\PY{n}{Vd}
\PY{k}{print}\PY{p}{(}\PY{l+s}{\PYZdq{}}\PY{l+s}{El valor estimado es de:\PYZob{}0\PYZcb{}}\PY{l+s}{\PYZdq{}}\PY{o}{.}\PY{n}{format}\PY{p}{(}\PY{n}{yp}\PY{p}{)}\PY{p}{)}
\end{Verbatim}

            
                \vspace{-0.2\baselineskip}
            
        \end{ColorVerbatim}
    

    

        % If the first block is an image, minipage the image.  Else
        % request a certain amount of space for the input text.
        \needspace{4\baselineskip}
        
        

            % Add document contents.
            
                \begin{InvisibleVerbatim}
                \vspace{-0.5\baselineskip}
\begin{alltt}El valor estimado es de:[ 5.00009139]
\end{alltt}

            \end{InvisibleVerbatim}
            
        
    


    % Make sure that atleast 4 lines are below the HR
    \needspace{4\baselineskip}

    
        \vspace{6pt}
        \makebox[0.1\linewidth]{\smaller\hfill\tt\color{nbframe-in-prompt}In\hspace{4pt}{[}6{]}:\hspace{4pt}}\\*
        \vspace{-2.65\baselineskip}
        \begin{ColorVerbatim}
            \vspace{-0.7\baselineskip}
            \begin{Verbatim}[commandchars=\\\{\}]

\end{Verbatim}

            
                \vspace{0.3\baselineskip}
            
        \end{ColorVerbatim}
    

        

        \renewcommand{\indexname}{Index}
        \printindex

    % End of document
    \end{document}


