
% Default to the notebook output style

    


% Inherit from the specified cell style.




    
\documentclass{article}

    
    
    \usepackage{graphicx} % Used to insert images
    \usepackage{adjustbox} % Used to constrain images to a maximum size 
    \usepackage{color} % Allow colors to be defined
    \usepackage{enumerate} % Needed for markdown enumerations to work
    \usepackage{geometry} % Used to adjust the document margins
    \usepackage{amsmath} % Equations
    \usepackage{amssymb} % Equations
    \usepackage[mathletters]{ucs} % Extended unicode (utf-8) support
    \usepackage[utf8x]{inputenc} % Allow utf-8 characters in the tex document
    \usepackage{fancyvrb} % verbatim replacement that allows latex
    \usepackage{grffile} % extends the file name processing of package graphics 
                         % to support a larger range 
    % The hyperref package gives us a pdf with properly built
    % internal navigation ('pdf bookmarks' for the table of contents,
    % internal cross-reference links, web links for URLs, etc.)
    \usepackage{hyperref}
    \usepackage{longtable} % longtable support required by pandoc >1.10
    

    
    
    \definecolor{orange}{cmyk}{0,0.4,0.8,0.2}
    \definecolor{darkorange}{rgb}{.71,0.21,0.01}
    \definecolor{darkgreen}{rgb}{.12,.54,.11}
    \definecolor{myteal}{rgb}{.26, .44, .56}
    \definecolor{gray}{gray}{0.45}
    \definecolor{lightgray}{gray}{.95}
    \definecolor{mediumgray}{gray}{.8}
    \definecolor{inputbackground}{rgb}{.95, .95, .85}
    \definecolor{outputbackground}{rgb}{.95, .95, .95}
    \definecolor{traceback}{rgb}{1, .95, .95}
    % ansi colors
    \definecolor{red}{rgb}{.6,0,0}
    \definecolor{green}{rgb}{0,.65,0}
    \definecolor{brown}{rgb}{0.6,0.6,0}
    \definecolor{blue}{rgb}{0,.145,.698}
    \definecolor{purple}{rgb}{.698,.145,.698}
    \definecolor{cyan}{rgb}{0,.698,.698}
    \definecolor{lightgray}{gray}{0.5}
    
    % bright ansi colors
    \definecolor{darkgray}{gray}{0.25}
    \definecolor{lightred}{rgb}{1.0,0.39,0.28}
    \definecolor{lightgreen}{rgb}{0.48,0.99,0.0}
    \definecolor{lightblue}{rgb}{0.53,0.81,0.92}
    \definecolor{lightpurple}{rgb}{0.87,0.63,0.87}
    \definecolor{lightcyan}{rgb}{0.5,1.0,0.83}
    
    % commands and environments needed by pandoc snippets
    % extracted from the output of `pandoc -s`
    \DefineVerbatimEnvironment{Highlighting}{Verbatim}{commandchars=\\\{\}}
    % Add ',fontsize=\small' for more characters per line
    \newenvironment{Shaded}{}{}
    \newcommand{\KeywordTok}[1]{\textcolor[rgb]{0.00,0.44,0.13}{\textbf{{#1}}}}
    \newcommand{\DataTypeTok}[1]{\textcolor[rgb]{0.56,0.13,0.00}{{#1}}}
    \newcommand{\DecValTok}[1]{\textcolor[rgb]{0.25,0.63,0.44}{{#1}}}
    \newcommand{\BaseNTok}[1]{\textcolor[rgb]{0.25,0.63,0.44}{{#1}}}
    \newcommand{\FloatTok}[1]{\textcolor[rgb]{0.25,0.63,0.44}{{#1}}}
    \newcommand{\CharTok}[1]{\textcolor[rgb]{0.25,0.44,0.63}{{#1}}}
    \newcommand{\StringTok}[1]{\textcolor[rgb]{0.25,0.44,0.63}{{#1}}}
    \newcommand{\CommentTok}[1]{\textcolor[rgb]{0.38,0.63,0.69}{\textit{{#1}}}}
    \newcommand{\OtherTok}[1]{\textcolor[rgb]{0.00,0.44,0.13}{{#1}}}
    \newcommand{\AlertTok}[1]{\textcolor[rgb]{1.00,0.00,0.00}{\textbf{{#1}}}}
    \newcommand{\FunctionTok}[1]{\textcolor[rgb]{0.02,0.16,0.49}{{#1}}}
    \newcommand{\RegionMarkerTok}[1]{{#1}}
    \newcommand{\ErrorTok}[1]{\textcolor[rgb]{1.00,0.00,0.00}{\textbf{{#1}}}}
    \newcommand{\NormalTok}[1]{{#1}}
    
    % Define a nice break command that doesn't care if a line doesn't already
    % exist.
    \def\br{\hspace*{\fill} \\* }
    % Math Jax compatability definitions
    \def\gt{>}
    \def\lt{<}
    % Document parameters
    \title{Instrumentacion}
    
    
    

    % Pygments definitions
    
\makeatletter
\def\PY@reset{\let\PY@it=\relax \let\PY@bf=\relax%
    \let\PY@ul=\relax \let\PY@tc=\relax%
    \let\PY@bc=\relax \let\PY@ff=\relax}
\def\PY@tok#1{\csname PY@tok@#1\endcsname}
\def\PY@toks#1+{\ifx\relax#1\empty\else%
    \PY@tok{#1}\expandafter\PY@toks\fi}
\def\PY@do#1{\PY@bc{\PY@tc{\PY@ul{%
    \PY@it{\PY@bf{\PY@ff{#1}}}}}}}
\def\PY#1#2{\PY@reset\PY@toks#1+\relax+\PY@do{#2}}

\expandafter\def\csname PY@tok@k\endcsname{\let\PY@bf=\textbf\def\PY@tc##1{\textcolor[rgb]{0.00,0.50,0.00}{##1}}}
\expandafter\def\csname PY@tok@cp\endcsname{\def\PY@tc##1{\textcolor[rgb]{0.74,0.48,0.00}{##1}}}
\expandafter\def\csname PY@tok@kc\endcsname{\let\PY@bf=\textbf\def\PY@tc##1{\textcolor[rgb]{0.00,0.50,0.00}{##1}}}
\expandafter\def\csname PY@tok@nd\endcsname{\def\PY@tc##1{\textcolor[rgb]{0.67,0.13,1.00}{##1}}}
\expandafter\def\csname PY@tok@sb\endcsname{\def\PY@tc##1{\textcolor[rgb]{0.73,0.13,0.13}{##1}}}
\expandafter\def\csname PY@tok@mo\endcsname{\def\PY@tc##1{\textcolor[rgb]{0.40,0.40,0.40}{##1}}}
\expandafter\def\csname PY@tok@na\endcsname{\def\PY@tc##1{\textcolor[rgb]{0.49,0.56,0.16}{##1}}}
\expandafter\def\csname PY@tok@err\endcsname{\def\PY@bc##1{\setlength{\fboxsep}{0pt}\fcolorbox[rgb]{1.00,0.00,0.00}{1,1,1}{\strut ##1}}}
\expandafter\def\csname PY@tok@kt\endcsname{\def\PY@tc##1{\textcolor[rgb]{0.69,0.00,0.25}{##1}}}
\expandafter\def\csname PY@tok@se\endcsname{\let\PY@bf=\textbf\def\PY@tc##1{\textcolor[rgb]{0.73,0.40,0.13}{##1}}}
\expandafter\def\csname PY@tok@sx\endcsname{\def\PY@tc##1{\textcolor[rgb]{0.00,0.50,0.00}{##1}}}
\expandafter\def\csname PY@tok@c1\endcsname{\let\PY@it=\textit\def\PY@tc##1{\textcolor[rgb]{0.25,0.50,0.50}{##1}}}
\expandafter\def\csname PY@tok@nl\endcsname{\def\PY@tc##1{\textcolor[rgb]{0.63,0.63,0.00}{##1}}}
\expandafter\def\csname PY@tok@ge\endcsname{\let\PY@it=\textit}
\expandafter\def\csname PY@tok@no\endcsname{\def\PY@tc##1{\textcolor[rgb]{0.53,0.00,0.00}{##1}}}
\expandafter\def\csname PY@tok@go\endcsname{\def\PY@tc##1{\textcolor[rgb]{0.53,0.53,0.53}{##1}}}
\expandafter\def\csname PY@tok@nv\endcsname{\def\PY@tc##1{\textcolor[rgb]{0.10,0.09,0.49}{##1}}}
\expandafter\def\csname PY@tok@vg\endcsname{\def\PY@tc##1{\textcolor[rgb]{0.10,0.09,0.49}{##1}}}
\expandafter\def\csname PY@tok@gu\endcsname{\let\PY@bf=\textbf\def\PY@tc##1{\textcolor[rgb]{0.50,0.00,0.50}{##1}}}
\expandafter\def\csname PY@tok@s1\endcsname{\def\PY@tc##1{\textcolor[rgb]{0.73,0.13,0.13}{##1}}}
\expandafter\def\csname PY@tok@sh\endcsname{\def\PY@tc##1{\textcolor[rgb]{0.73,0.13,0.13}{##1}}}
\expandafter\def\csname PY@tok@w\endcsname{\def\PY@tc##1{\textcolor[rgb]{0.73,0.73,0.73}{##1}}}
\expandafter\def\csname PY@tok@mf\endcsname{\def\PY@tc##1{\textcolor[rgb]{0.40,0.40,0.40}{##1}}}
\expandafter\def\csname PY@tok@bp\endcsname{\def\PY@tc##1{\textcolor[rgb]{0.00,0.50,0.00}{##1}}}
\expandafter\def\csname PY@tok@cm\endcsname{\let\PY@it=\textit\def\PY@tc##1{\textcolor[rgb]{0.25,0.50,0.50}{##1}}}
\expandafter\def\csname PY@tok@nn\endcsname{\let\PY@bf=\textbf\def\PY@tc##1{\textcolor[rgb]{0.00,0.00,1.00}{##1}}}
\expandafter\def\csname PY@tok@kd\endcsname{\let\PY@bf=\textbf\def\PY@tc##1{\textcolor[rgb]{0.00,0.50,0.00}{##1}}}
\expandafter\def\csname PY@tok@gr\endcsname{\def\PY@tc##1{\textcolor[rgb]{1.00,0.00,0.00}{##1}}}
\expandafter\def\csname PY@tok@m\endcsname{\def\PY@tc##1{\textcolor[rgb]{0.40,0.40,0.40}{##1}}}
\expandafter\def\csname PY@tok@s2\endcsname{\def\PY@tc##1{\textcolor[rgb]{0.73,0.13,0.13}{##1}}}
\expandafter\def\csname PY@tok@ow\endcsname{\let\PY@bf=\textbf\def\PY@tc##1{\textcolor[rgb]{0.67,0.13,1.00}{##1}}}
\expandafter\def\csname PY@tok@cs\endcsname{\let\PY@it=\textit\def\PY@tc##1{\textcolor[rgb]{0.25,0.50,0.50}{##1}}}
\expandafter\def\csname PY@tok@gt\endcsname{\def\PY@tc##1{\textcolor[rgb]{0.00,0.27,0.87}{##1}}}
\expandafter\def\csname PY@tok@ss\endcsname{\def\PY@tc##1{\textcolor[rgb]{0.10,0.09,0.49}{##1}}}
\expandafter\def\csname PY@tok@c\endcsname{\let\PY@it=\textit\def\PY@tc##1{\textcolor[rgb]{0.25,0.50,0.50}{##1}}}
\expandafter\def\csname PY@tok@gi\endcsname{\def\PY@tc##1{\textcolor[rgb]{0.00,0.63,0.00}{##1}}}
\expandafter\def\csname PY@tok@ne\endcsname{\let\PY@bf=\textbf\def\PY@tc##1{\textcolor[rgb]{0.82,0.25,0.23}{##1}}}
\expandafter\def\csname PY@tok@o\endcsname{\def\PY@tc##1{\textcolor[rgb]{0.40,0.40,0.40}{##1}}}
\expandafter\def\csname PY@tok@mi\endcsname{\def\PY@tc##1{\textcolor[rgb]{0.40,0.40,0.40}{##1}}}
\expandafter\def\csname PY@tok@s\endcsname{\def\PY@tc##1{\textcolor[rgb]{0.73,0.13,0.13}{##1}}}
\expandafter\def\csname PY@tok@vc\endcsname{\def\PY@tc##1{\textcolor[rgb]{0.10,0.09,0.49}{##1}}}
\expandafter\def\csname PY@tok@nb\endcsname{\def\PY@tc##1{\textcolor[rgb]{0.00,0.50,0.00}{##1}}}
\expandafter\def\csname PY@tok@gp\endcsname{\let\PY@bf=\textbf\def\PY@tc##1{\textcolor[rgb]{0.00,0.00,0.50}{##1}}}
\expandafter\def\csname PY@tok@sr\endcsname{\def\PY@tc##1{\textcolor[rgb]{0.73,0.40,0.53}{##1}}}
\expandafter\def\csname PY@tok@gs\endcsname{\let\PY@bf=\textbf}
\expandafter\def\csname PY@tok@gh\endcsname{\let\PY@bf=\textbf\def\PY@tc##1{\textcolor[rgb]{0.00,0.00,0.50}{##1}}}
\expandafter\def\csname PY@tok@si\endcsname{\let\PY@bf=\textbf\def\PY@tc##1{\textcolor[rgb]{0.73,0.40,0.53}{##1}}}
\expandafter\def\csname PY@tok@sc\endcsname{\def\PY@tc##1{\textcolor[rgb]{0.73,0.13,0.13}{##1}}}
\expandafter\def\csname PY@tok@kp\endcsname{\def\PY@tc##1{\textcolor[rgb]{0.00,0.50,0.00}{##1}}}
\expandafter\def\csname PY@tok@sd\endcsname{\let\PY@it=\textit\def\PY@tc##1{\textcolor[rgb]{0.73,0.13,0.13}{##1}}}
\expandafter\def\csname PY@tok@mh\endcsname{\def\PY@tc##1{\textcolor[rgb]{0.40,0.40,0.40}{##1}}}
\expandafter\def\csname PY@tok@vi\endcsname{\def\PY@tc##1{\textcolor[rgb]{0.10,0.09,0.49}{##1}}}
\expandafter\def\csname PY@tok@kn\endcsname{\let\PY@bf=\textbf\def\PY@tc##1{\textcolor[rgb]{0.00,0.50,0.00}{##1}}}
\expandafter\def\csname PY@tok@kr\endcsname{\let\PY@bf=\textbf\def\PY@tc##1{\textcolor[rgb]{0.00,0.50,0.00}{##1}}}
\expandafter\def\csname PY@tok@ni\endcsname{\let\PY@bf=\textbf\def\PY@tc##1{\textcolor[rgb]{0.60,0.60,0.60}{##1}}}
\expandafter\def\csname PY@tok@nc\endcsname{\let\PY@bf=\textbf\def\PY@tc##1{\textcolor[rgb]{0.00,0.00,1.00}{##1}}}
\expandafter\def\csname PY@tok@nt\endcsname{\let\PY@bf=\textbf\def\PY@tc##1{\textcolor[rgb]{0.00,0.50,0.00}{##1}}}
\expandafter\def\csname PY@tok@gd\endcsname{\def\PY@tc##1{\textcolor[rgb]{0.63,0.00,0.00}{##1}}}
\expandafter\def\csname PY@tok@il\endcsname{\def\PY@tc##1{\textcolor[rgb]{0.40,0.40,0.40}{##1}}}
\expandafter\def\csname PY@tok@nf\endcsname{\def\PY@tc##1{\textcolor[rgb]{0.00,0.00,1.00}{##1}}}

\def\PYZbs{\char`\\}
\def\PYZus{\char`\_}
\def\PYZob{\char`\{}
\def\PYZcb{\char`\}}
\def\PYZca{\char`\^}
\def\PYZam{\char`\&}
\def\PYZlt{\char`\<}
\def\PYZgt{\char`\>}
\def\PYZsh{\char`\#}
\def\PYZpc{\char`\%}
\def\PYZdl{\char`\$}
\def\PYZhy{\char`\-}
\def\PYZsq{\char`\'}
\def\PYZdq{\char`\"}
\def\PYZti{\char`\~}
% for compatibility with earlier versions
\def\PYZat{@}
\def\PYZlb{[}
\def\PYZrb{]}
\makeatother


    % Exact colors from NB
    \definecolor{incolor}{rgb}{0.0, 0.0, 0.5}
    \definecolor{outcolor}{rgb}{0.545, 0.0, 0.0}



    
    % Prevent overflowing lines due to hard-to-break entities
    \sloppy 
    % Setup hyperref package
    \hypersetup{
      breaklinks=true,  % so long urls are correctly broken across lines
      colorlinks=true,
      urlcolor=blue,
      linkcolor=darkorange,
      citecolor=darkgreen,
      }
    % Slightly bigger margins than the latex defaults
    
    \geometry{verbose,tmargin=1in,bmargin=1in,lmargin=1in,rmargin=1in}
    
    

    \begin{document}
    
    
    \maketitle
    
    

    

    \section{Introducción a la obtención de curvas}


    \section{Introducción}\label{introducciuxf3n}

El desarrollo de un instrumento de medición es un proceso que parte del
hecho que podemos asociar una variable del enterno a una variable
eléctrica (Voltaje, Corriente, Frecuencia, Fase). Sin embargo, en el
proceso de medición siempre se introducen errores e incertidumbres, las
cuales no necesariamente de pueden evitar. Esto hace necesario el
proceso de calibración, el cual consiste en garantizar que el proceso de
medición se ajuste a un patron de referencia.

Por lo que el primer paso en el diseño de cualquier sistema de medición
es caracterizar los sensores a utilizar, a través de un sistema de
referencia que necesita estar certificado con algún patron de medición.

\subsubsection{Curva base}\label{curva-base}

La gran mayoria de fabricantes proporcionan una curva básica de ajuste
para sus sensores de forma que si un sensor se dice lineal, entonces se
espera que tenga la siguiente expresión:

\begin{equation}
y=ax+b
\end{equation}

Entonces basta concer los valores de $a$ y $b$ para poder conocer el
valor de $y$ a partir del valor de $x$. Sin embargo, en la practica por
errores de construcción o incertidumbres parametricas esto no sucede ya
que la ecuación exacta no se conoce o resulta muy dificil conocer por lo
que normalmente se utiliza la tecnica de \textbf{ajusta a una curva}
para poder obtener una la expresión:

\begin{equation}
y=\hat{a}x+\hat{b}+\varepsilon
\end{equation}

donde $\hat{a}$ y $\hat{b}$ son obtenidos a través de una aproximación
númerica, $\varepsilon$ es un error que no puede ser evitado y el cual
solo se estima.

    \begin{Verbatim}[commandchars=\\\{\}]
{\color{incolor}In [{\color{incolor}74}]:} \PY{k+kn}{import} \PY{n+nn}{numpy} \PY{k+kn}{as} \PY{n+nn}{np}
         \PY{k+kn}{import} \PY{n+nn}{scipy}
         
         \PY{o}{\PYZpc{}}\PY{k}{pylab} \PY{n}{inline}
         
         \PY{k+kn}{from} \PY{n+nn}{matplotlib} \PY{k+kn}{import} \PY{n}{pylab}
         
         \PY{c}{\PYZsh{}Linear regression example}
         \PY{c}{\PYZsh{} This is a very simple example of using two scipy tools }
         \PY{c}{\PYZsh{} for linear regression, polyfit and stats.linregress}
         
         \PY{c}{\PYZsh{}Sample data creation}
         \PY{c}{\PYZsh{}number of points }
         \PY{n}{n}\PY{o}{=}\PY{l+m+mi}{50}
         \PY{n}{t}\PY{o}{=}\PY{n}{linspace}\PY{p}{(}\PY{o}{\PYZhy{}}\PY{l+m+mi}{5}\PY{p}{,}\PY{l+m+mi}{5}\PY{p}{,}\PY{n}{n}\PY{p}{)}
         \PY{c}{\PYZsh{}parameters}
         \PY{n}{a}\PY{o}{=}\PY{l+m+mf}{0.8}\PY{p}{;} \PY{n}{b}\PY{o}{=}\PY{o}{\PYZhy{}}\PY{l+m+mi}{4}
         \PY{n}{fx}\PY{o}{=}\PY{n}{poly1d}
         \PY{n}{x}\PY{o}{=}\PY{n}{polyval}\PY{p}{(}\PY{p}{[}\PY{n}{a}\PY{p}{,}\PY{n}{b}\PY{p}{]}\PY{p}{,}\PY{n}{t}\PY{p}{)}
         \PY{c}{\PYZsh{}add some noise}
         \PY{n}{xn}\PY{o}{=}\PY{n}{x}\PY{o}{+}\PY{n}{randn}\PY{p}{(}\PY{n}{n}\PY{p}{)}
         \PY{c}{\PYZsh{}matplotlib ploting}
         \PY{n}{matplotlib}\PY{o}{.}\PY{n}{rcParams}\PY{o}{.}\PY{n}{update}\PY{p}{(}\PY{p}{\PYZob{}}\PY{l+s}{\PYZsq{}}\PY{l+s}{font.size}\PY{l+s}{\PYZsq{}}\PY{p}{:} \PY{l+m+mi}{18}\PY{p}{\PYZcb{}}\PY{p}{)}
         \PY{n}{figure}\PY{p}{(}\PY{n}{figsize}\PY{o}{=}\PY{p}{(}\PY{l+m+mi}{14}\PY{p}{,}\PY{l+m+mi}{10}\PY{p}{)}\PY{p}{,} \PY{n}{dpi}\PY{o}{=}\PY{l+m+mi}{150}\PY{p}{)}
         \PY{n}{xlabel}\PY{p}{(}\PY{l+s}{\PYZsq{}}\PY{l+s}{Variable Independiente}\PY{l+s}{\PYZsq{}}\PY{p}{)}
         \PY{n}{ylabel}\PY{p}{(}\PY{l+s}{\PYZsq{}}\PY{l+s}{Variable Dependiente}\PY{l+s}{\PYZsq{}}\PY{p}{)}
         \PY{n}{plot}\PY{p}{(}\PY{n}{t}\PY{p}{,}\PY{n}{x}\PY{p}{)}
         \PY{n}{plot}\PY{p}{(}\PY{n}{t}\PY{p}{,}\PY{n}{xn}\PY{p}{,}\PY{l+s}{\PYZsq{}}\PY{l+s}{r*}\PY{l+s}{\PYZsq{}}\PY{p}{,}\PY{n}{markersize}\PY{o}{=}\PY{l+m+mi}{13}\PY{p}{)}
         \PY{n}{plot}\PY{p}{(}\PY{p}{[}\PY{n}{t}\PY{p}{[}\PY{l+m+mi}{10}\PY{p}{]}\PY{p}{,}\PY{n}{t}\PY{p}{[}\PY{l+m+mi}{10}\PY{p}{]}\PY{p}{]}\PY{p}{,}\PY{p}{[}\PY{n}{x}\PY{p}{[}\PY{l+m+mi}{10}\PY{p}{]}\PY{p}{,}\PY{n}{xn}\PY{p}{[}\PY{l+m+mi}{10}\PY{p}{]}\PY{p}{]}\PY{p}{,}\PY{l+s}{\PYZsq{}}\PY{l+s}{g}\PY{l+s}{\PYZsq{}}\PY{p}{,}\PY{n}{linewidth}\PY{o}{=}\PY{l+m+mi}{3}\PY{p}{)}
         \PY{n}{grid}\PY{p}{(}\PY{p}{)}
\end{Verbatim}

    \begin{Verbatim}[commandchars=\\\{\}]
Populating the interactive namespace from numpy and matplotlib
    \end{Verbatim}

    \begin{center}
    \adjustimage{max size={0.9\linewidth}{0.9\paperheight}}{Instrumentacion_files/Instrumentacion_2_1.png}
    \end{center}
    { \hspace*{\fill} \\}
    
    \subsection{\#\#Regresion Lineal}\label{regresion-lineal}

En estadística la regresión lineal o ajuste lineal es un método
matemático que modela la relación entre una variable dependiente $Y$,
las variables independientes $X_i$ y un término aleatorio $\varepsilon$.
Este modelo puede ser expresado como: \[
Y_t = \beta_0  + \beta_1 X_1 + \beta_2 X_2 +  \cdots +\beta_p X_p + \varepsilon
\]

$Y_t$: variable dependiente, explicada o regresando.

$X_1$, $X_2$, $\cdots, X_p$ : variables explicativas, independientes o
regresores.

$\beta_0$,$\beta_1$,$\beta_2$,$\cdots$ ,$\beta_p$ : parámetros, miden la
influencia que las variables explicativas tienen sobre el regresando.

donde $\beta_0$ es la intersección o término ``constante'', las
$\beta_i \ (i > 0)$ son los parámetros respectivos a cada variable
independiente, y $p$ es el número de parámetros independientes a tener
en cuenta en la regresión. La regresión lineal puede ser contrastada con
la regresión no lineal.

\subsubsection{El modelo de regresión
lineal}\label{el-modelo-de-regresiuxf3n-lineal}

El modelo lineal relaciona la variable dependiente $Y$ con $K$ variables
explicativas $X_k \ (k = 1,...K)$, o cualquier transformación de éstas,
que generan un hiperplano de parámetros \beta\_k desconocidos:

\begin{equation}
 Y = \sum \beta_k X_k + \varepsilon
\end{equation}

donde \varepsilon  es la perturbación aleatoria que recoge todos
aquellos factores de la realidad no controlables u observables y que por
tanto se asocian con el azar, y es la que confiere al modelo su carácter
estocástico. En el caso más sencillo, con una sola variable explicativa,
el hiperplano es una recta:

\begin{equation}
Y = \beta_1 + \beta_2 X_2 + \varepsilon
\end{equation}

El problema de la regresión consiste en elegir unos valores determinados
para los parámetros desconocidos $\beta_k$, de modo que la ecuación
quede completamente especificada. Para ello se necesita un conjunto de
observaciones. En una observación cualquiera i-ésima $(i= 1,... I)$ se
registra el comportamiento simultáneo de la variable dependiente y las
variables explicativas (las perturbaciones aleatorias se suponen no
observables).

\begin{equation}
Y_i = \sum \beta_k X_{ki} + \varepsilon_i
\end{equation}

Los valores escogidos como estimadores de los parámetros,
$\hat{\beta_k}$, son los coeficientes de regresión, sin que se pueda
garantizar que coinciden con parámetros reales del proceso generador.
Por tanto, en

\begin{equation}
Y_i = \sum \hat{\beta_k} X_{ki} + \hat{\varepsilon_i}
\end{equation}

Los valores \hat{\varepsilon_i} son por su parte estimaciones de la
perturbación aleatoria o errores.

\subsubsection{Hipótesis modelo de regresión lineal
clásico}\label{hipuxf3tesis-modelo-de-regresiuxf3n-lineal-cluxe1sico}

\begin{enumerate}
\def\labelenumi{\arabic{enumi}.}
\item
  Esperanza matemática nula. \[E(\varepsilon_i) = 0\] Para cada valor de
  X la perturbación tomará distintos valores de forma aleatoria, pero no
  tomará sistemáticamente valores positivos o negativos, sino que se
  supone que tomará algunos valores mayores que cero y otros menores, de
  tal forma que su valor esperado sea cero.
\item
  Homocedasticidad
  \[Var(\varepsilon_t) = E(\varepsilon_t - E \varepsilon_t)^2 = E \varepsilon_t^2 = \sigma^2\]
  para todo t. Todos los términos de la perturbación tienen la misma
  varianza que es desconocida. La dispersión de cada \varepsilon\_t en
  torno a su valor esperado es siempre la misma.
\item
  Incorrelación.
  \[Cov(\varepsilon_t,\varepsilon_s ) = (\varepsilon_t - E \varepsilon_t) (\varepsilon_s - E \varepsilon_s) = E \varepsilon_t \varepsilon_s = 0 \]
  para todo $t,s$ con $t$ distinto de $s$. Las covarianzas entre las
  distintas pertubaciones son nulas, lo que quiere decir que no están
  correlacionadas o autocorrelacionadas. Esto implica que el valor de la
  perturbación para cualquier observación muestral no viene influenciado
  por los valores de la perturbación correspondientes a otras
  observaciones muestrales.
\item
  Regresores no estocásticos.
\item
  No existen relaciones lineales exactas entre los regresores.
\item
  $T > k + 1$ Suponemos que no existen errores de especificación en el
  modelo ni errores de medida en las variables explicativas
\item
  Normalidad de las perturbaciones $\varepsilon -> N(0, \sigma^2 )$
\end{enumerate}

\subsubsection{Supuestos del modelo de regresión
lineal}\label{supuestos-del-modelo-de-regresiuxf3n-lineal}

Para poder crear un modelo de regresión lineal, es necesario que se
cumpla con los siguientes supuestos:3

\emph{La relación entre las variables es lineal. }Los errores en la
medición de las variables explicativas son independientes entre sí.
\emph{Los errores tienen varianza constante. (Homocedasticidad) }Los
errores tienen una esperanza matemática igual a cero (los errores de una
misma magnitud y distinto signo son equiprobables). *El error total es
la suma de todos los errores.

\subsubsection{Tipos de modelos de regresión
lineal}\label{tipos-de-modelos-de-regresiuxf3n-lineal}

Existen diferentes tipos de regresión lineal que se clasifican de
acuerdo a sus parámetros:

\paragraph{Regresión lineal simple}\label{regresiuxf3n-lineal-simple}

Sólo se maneja una variable independiente, por lo que sólo cuenta con
dos parámetros. Son de la forma:4

\begin{equation}
Y_i = \beta_0 + \beta_1 X_i + \varepsilon_i
\end{equation}

donde $\varepsilon_i$ es el error asociado a la medición del valor $X_i$
y siguen los supuestos de modo que $\varepsilon_i$ $\sim N(0,\sigma^2)$
(media cero, varianza constante e igual a un $\sigma$ y
$\varepsilon_i \perp \varepsilon_j$ con $i\neq j$).

\subsubsection{Análisis}\label{anuxe1lisis}

Dado el modelo de regresión simple, si se calcula la esperanza (valor
esperado) del valor Y, se obtiene:5

\begin{equation}
E(y_i) = \hat{y_i}=E(\beta_0) + E(\beta_1 x_i) + E(\varepsilon_i)
\end{equation}

Derivando respecto a $\hat{\beta}_0$ y $\hat{\beta}_1$ e igualando a
cero, se obtiene:

\begin{equation}
\frac{\partial \sum (y_i - \hat{y_i})^2 }{\partial \hat{\beta}_0} = 0 
\end{equation}

\begin{equation}
\frac{\partial \sum (y_i - \hat{y_i})^2 }{\partial \hat{\beta}_1} = 0 
\end{equation}

Obteniendo dos ecuaciones denominadas ecuaciones normales que generan la
siguiente solución para ambos parámetros:4

\begin{equation}
\hat{\beta_1} = \frac { \sum x \sum y - n \sum xy } { \left ( \sum x \right ) ^ 2 - n \sum x^2 } = \frac{ \sum (x-\bar{x})(y-\bar{y} ) }{\sum ( x - \bar{x})^2 }
\end{equation}

\begin{equation}
\hat{\beta_0} = \frac { \sum y - \hat{\beta}_1 \sum x } { n } = \bar{y} - \hat{\beta_1} \bar{x}
\end{equation}

La interpretación del parámetro \{$\beta_1$\} es que un incremento en
$Xi$ de una unidad, $Yi$ incrementará en \{$\beta_1$\}

\subsubsection{Regresión lineal
múltiple}\label{regresiuxf3n-lineal-muxfaltiple}

La regresion lineal nos permite trabajar con una variable a nivel de
intervalo o razón, así también se puede comprender la relación de dos o
más variables y nos permitirá relacionar mediante ecuaciones, una
variable en relación a otras variables llamándose Regresión múltiple.
Constantemente en la práctica de la investigación estadística, se
encuentran variables que de alguna manera están relacionados entre si,
por lo que es posible que una de las variables puedan relacionarse
matemáticamente en función de otra u otras variables.

Maneja varias variables independientes. Cuenta con varios parámetros. Se
expresan de la forma:6

\begin{equation}
Y_i = \beta_0 + \sum \beta_i X_{ip} + \varepsilon_i
\end{equation}

donde $\varepsilon_i$ es el error asociado a la medición $i$ del valor
$X_{ip}$ y siguen los supuestos de modo que
$\varepsilon_i \sim N(0,\sigma^2)$ (media cero, varianza constante e
igual a un $\sigma$ y $\varepsilon_i \perp \varepsilon_j$ con
$i\neq j$).

\subsubsection{Rectas de regresión}\label{rectas-de-regresiuxf3n}

Las rectas de regresión son las rectas que mejor se ajustan a la nube de
puntos (o también llamado diagrama de dispersión) generada por una
distribución binomial. Matemáticamente, son posibles dos rectas de
máximo ajuste:7

La recta de regresión de $Y$ sobre $X$:

\begin{equation}
y = \bar{y} + \frac{\sigma_{xy}}{\sigma_{x}^2}(x - \bar{x})
\end{equation}

La recta de regresión de $X$ sobre $Y$:

\begin{equation}
x = \bar{x} + \frac{\sigma_{xy}}{\sigma_{y}^2}(y - \bar{y})
\end{equation}

La correlación (``$r$'') de las rectas determinará la calidad del
ajuste. Si $r$ es cercano o igual a 1, el ajuste será bueno y las
predicciones realizadas a partir del modelo obtenido serán muy fiables
(el modelo obtenido resulta verdaderamente representativo); si $r$ es
cercano o igual a 0, se tratará de un ajuste malo en el que las
predicciones que se realicen a partir del modelo obtenido no serán
fiables (el modelo obtenido no resulta representativo de la realidad).
Ambas rectas de regresión se intersecan en un punto llamado centro de
gravedad de la distribución.


    \subsection{Ajuste de curvas}


    El ajuste de curvas consiste en encontrar una curva que contenga una
serie de puntos y que posiblemente cumpla una serie de restricciones
adicionales. Esta sección es una introducción tanto a la interpolación
(cuando se espera un ajuste exacto a determinadas restricciones) y al
ajuste de curvas/análisis de regresión (cuando se permite una
aproximación).

    Ajuste de líneas y curvas polinómicas a puntos{[}editar · editar
fuente{]}

Empecemos con una ecuación polinómica de primer grado:

\[
y = ax + b\;
\]

Esta línea tiene pendiente a. Sabemos que habrá una línea conectando dos
puntos cualesquiera. Por tanto, una ecuación polinómica de primer grado
es un ajuste perfecto entre dos puntos.

Si aumentamos el orden de la ecuación a la de un polinomio de segundo
grado, obtenemos:

\[
y = ax^2 + bx + c\;
\]

Esto se ajustará exactamente a tres puntos. Si aumentamos el orden de la
ecuación a la de un polinomio de tercer grado, obtenemos:

\[
y = ax^3 + bx^2 + cx + d\;
\]

que se ajustará a cuatro puntos.

Una forma más general de decirlo es que se ajustará exactamente a cuatro
restricciones. Cada restricción puede ser un punto, un ángulo o una
curvatura (que es el recíproco del radio, o $1/R$). Las restricciones de
ángulo y curvatura se suelen añadir a los extremos de una curva, y en
tales casos se les llama condiciones finales. A menudo se usan
condiciones finales idénticas para asegurar una transición suave entre
curvas polinómicas contenidas en una única spline. También se pueden
añadir restricciones de orden alto, como ``el cambio en la tasa de
curvatura''. Esto, por ejemplo, sería útil en diseños de intercambios en
trébol para incorporaciones a autopistas, para entender las fuerzas a
las que somete a un vehículo y poder establecer límites razonables de
velocidad.

Si tenemos más de $n + 1$ restricciones (siendo $n$ el grado del
polinomio), aún podemos hacer pasar la curva polinómica por ellas. No es
seguro que vaya a existir un ajuste exacto a todas ellas (pero podría
suceder, por ejemplo, en el caso de un polinomio de primer grado que se
ajusta a tres puntos colineales). En general, sin embargo, se necesita
algún método para evaluar cada aproximación. El método de mínimos
cuadrados es una manera de comparar las desviaciones.

Ahora bien, podríamos preguntarnos la razón de querer un ajuste
aproximado cuando podríamos simplemente aumentar el grado de la ecuación
polinómica para obtener un ajuste exacto. Existen varias:

Incluso si existe un ajuste exacto, no quiere decir necesariamente que
podamos encontrarlo. Dependiendo del algoritmo que se use, podríamos
encontrar un caso divergente, donde no se podría calcular el ajuste
exacto, o el coste computacional de encontrar la solución podría ser muy
alto. De cualquier modo, tendríamos que acabar aceptando una solución
aproximada.Quizá prefiramos el efecto de promediar datos cuestionables
en una muestra, en lugar de distorsionar la curva para que se ajuste a
ellos de forma exacta.

Los polinomios de orden superior pueden oscilar mucho. Si hacemos pasar
una curva por los puntos $A$ y $B$, esperaríamos que la curva pase
también cerca del punto medio entre $A$ y $B$. Esto puede no suceder con
curvas polinómicas de grados altos, ya que pueden tener valores de
magnitud positiva o negativa muy grande. Con polinomios de grado bajo
existen más posibilidades de que la curva pase cerca del punto medio (y
queda garantizado que pasará exactamente por ahí, en los de primer
grado). Los polinomios de orden bajo tienden a ser suaves y las curvas
de los polinomios de orden alto tienden a ser ``bulbosas''. Para definir
esto con más precisión, el número máximo de puntos de inflexión de una
curva polinómica es $n-2$, donde $n$ es el orden de la ecuación
polinómica. Un punto de inflexión es el lugar de una curva donde cambia
de radio positivo a negativo. Obsérvese que la ``bulbosidad'' de los
polinomios de orden alto es sólo una posibilidad, ya que también pueden
ser suaves, pero no existen garantías, al contrario que sucede con los
polinomios de orden bajo. Un polinomio de grado quince podría tener,
como máximo, trece puntos de inflexión, pero podría tener también doce,
once, o cualquier número hasta cero.

Ahora que hemos hablado del uso de grados demasiado bajos para conseguir
un ajuste exacto, comentemos qué sucede si el grado de una curva
polinómica es mayor del necesario para dicho ajuste. Esto es malo por
las razones comentadas anteriormente si los polinomios son de orden
alto, pero también nos lleva a un caso en que exista un número infinito
de soluciones. Por ejemplo, un polinomio de primer grado (una línea)
restringido por un único punto, en lugar de los dos habituales, nos dará
un número infinito de soluciones. Esto nos trae el problema de cómo
comparar y escoger una solución única, lo que puede ser un problema
tanto para humanos como para el software. Por esta razón es mejor
escoger el polinomio de menor grado posible para obtener un ajuste
exacto en todas las restricciones, y quizá incluso un grado menor si es
aceptable una aproximación al ajuste.

    \begin{Verbatim}[commandchars=\\\{\}]
{\color{incolor}In [{\color{incolor}75}]:} \PY{k+kn}{from} \PY{n+nn}{scipy} \PY{k+kn}{import} \PY{n}{stats}
         \PY{c}{\PYZsh{}Linear regressison \PYZhy{}polyfit \PYZhy{} polyfit can be used other orders polys}
         \PY{p}{(}\PY{n}{ar}\PY{p}{,}\PY{n}{br}\PY{p}{)}\PY{o}{=}\PY{n}{polyfit}\PY{p}{(}\PY{n}{t}\PY{p}{,}\PY{n}{xn}\PY{p}{,}\PY{l+m+mi}{1}\PY{p}{)}
         \PY{n}{xr}\PY{o}{=}\PY{n}{polyval}\PY{p}{(}\PY{p}{[}\PY{n}{ar}\PY{p}{,}\PY{n}{br}\PY{p}{]}\PY{p}{,}\PY{n}{t}\PY{p}{)}
         \PY{c}{\PYZsh{}compute the mean square error}
         \PY{n}{err}\PY{o}{=}\PY{n}{sqrt}\PY{p}{(}\PY{n+nb}{sum}\PY{p}{(}\PY{p}{(}\PY{n}{xr}\PY{o}{\PYZhy{}}\PY{n}{xn}\PY{p}{)}\PY{o}{*}\PY{o}{*}\PY{l+m+mi}{2}\PY{p}{)}\PY{o}{/}\PY{n}{n}\PY{p}{)}
         
         \PY{c}{\PYZsh{}Linear \PYZsh{}2}
         \PY{n}{A} \PY{o}{=} \PY{n}{np}\PY{o}{.}\PY{n}{vstack}\PY{p}{(}\PY{p}{[}\PY{n}{t}\PY{p}{,} \PY{n}{np}\PY{o}{.}\PY{n}{ones}\PY{p}{(}\PY{n+nb}{len}\PY{p}{(}\PY{n}{t}\PY{p}{)}\PY{p}{)}\PY{p}{]}\PY{p}{)}\PY{o}{.}\PY{n}{T}
         \PY{n}{m}\PY{p}{,} \PY{n}{c} \PY{o}{=} \PY{n}{np}\PY{o}{.}\PY{n}{linalg}\PY{o}{.}\PY{n}{lstsq}\PY{p}{(}\PY{n}{A}\PY{p}{,} \PY{n}{xn}\PY{p}{)}\PY{p}{[}\PY{l+m+mi}{0}\PY{p}{]}
         \PY{k}{print}\PY{p}{(}\PY{n}{m}\PY{p}{,}\PY{n}{c}\PY{p}{)}
         \PY{n}{xr2}\PY{o}{=}\PY{n}{m}\PY{o}{*}\PY{n}{t}\PY{o}{+}\PY{n}{c}
         
         \PY{c}{\PYZsh{}Linear \PYZsh{}3}
         \PY{n}{slope}\PY{p}{,} \PY{n}{intercept}\PY{p}{,} \PY{n}{r\PYZus{}value}\PY{p}{,} \PY{n}{p\PYZus{}value}\PY{p}{,} \PY{n}{std\PYZus{}err} \PY{o}{=} \PY{n}{stats}\PY{o}{.}\PY{n}{linregress}\PY{p}{(}\PY{n}{t}\PY{p}{,}\PY{n}{xn}\PY{p}{)}
         \PY{n}{xr3}\PY{o}{=}\PY{n}{intercept} \PY{o}{+} \PY{n}{slope}\PY{o}{*}\PY{n}{t}
         
         \PY{k}{print}\PY{p}{(}\PY{l+s}{\PYZsq{}}\PY{l+s}{Linear regression using polyfit}\PY{l+s}{\PYZsq{}}\PY{p}{)}
         \PY{k}{print}\PY{p}{(}\PY{l+s}{\PYZsq{}}\PY{l+s}{parameters: a=}\PY{l+s+si}{\PYZpc{}.2f}\PY{l+s}{ b=}\PY{l+s+si}{\PYZpc{}.2f}\PY{l+s}{ }\PY{l+s+se}{\PYZbs{}n}\PY{l+s}{regression: a=}\PY{l+s+si}{\PYZpc{}.2f}\PY{l+s}{ b=}\PY{l+s+si}{\PYZpc{}.2f}\PY{l+s}{, ms error= }\PY{l+s+si}{\PYZpc{}.3f}\PY{l+s}{\PYZsq{}} \PY{o}{\PYZpc{}} \PY{p}{(}\PY{n}{a}\PY{p}{,}\PY{n}{b}\PY{p}{,}\PY{n}{ar}\PY{p}{,}\PY{n}{br}\PY{p}{,}\PY{n}{err}\PY{p}{)}\PY{p}{)}
         
         \PY{c}{\PYZsh{}matplotlib ploting}
         \PY{n}{matplotlib}\PY{o}{.}\PY{n}{rcParams}\PY{o}{.}\PY{n}{update}\PY{p}{(}\PY{p}{\PYZob{}}\PY{l+s}{\PYZsq{}}\PY{l+s}{font.size}\PY{l+s}{\PYZsq{}}\PY{p}{:} \PY{l+m+mi}{18}\PY{p}{\PYZcb{}}\PY{p}{)}
         \PY{n}{figure}\PY{p}{(}\PY{n}{figsize}\PY{o}{=}\PY{p}{(}\PY{l+m+mi}{14}\PY{p}{,}\PY{l+m+mi}{10}\PY{p}{)}\PY{p}{,} \PY{n}{dpi}\PY{o}{=}\PY{l+m+mi}{150}\PY{p}{)}
         \PY{n}{title}\PY{p}{(}\PY{l+s}{\PYZsq{}}\PY{l+s}{Linear Regression Example}\PY{l+s}{\PYZsq{}}\PY{p}{)}
         \PY{n}{plot}\PY{p}{(}\PY{n}{t}\PY{p}{,}\PY{n}{x}\PY{p}{,}\PY{l+s}{\PYZsq{}}\PY{l+s}{g.\PYZhy{}\PYZhy{}}\PY{l+s}{\PYZsq{}}\PY{p}{)}
         \PY{n}{plot}\PY{p}{(}\PY{n}{t}\PY{p}{,}\PY{n}{xn}\PY{p}{,}\PY{l+s}{\PYZsq{}}\PY{l+s}{k.}\PY{l+s}{\PYZsq{}}\PY{p}{)}
         \PY{n}{plot}\PY{p}{(}\PY{n}{t}\PY{p}{,}\PY{n}{xr}\PY{p}{,}\PY{l+s}{\PYZsq{}}\PY{l+s}{r.\PYZhy{}}\PY{l+s}{\PYZsq{}}\PY{p}{)}
         \PY{n}{plot}\PY{p}{(}\PY{n}{t}\PY{p}{,}\PY{n}{xr2}\PY{p}{,}\PY{l+s}{\PYZsq{}}\PY{l+s}{b\PYZhy{}\PYZhy{}}\PY{l+s}{\PYZsq{}}\PY{p}{)}
         \PY{n}{plot}\PY{p}{(}\PY{n}{t}\PY{p}{,}\PY{n}{xr3}\PY{p}{,}\PY{l+s}{\PYZsq{}}\PY{l+s}{yx}\PY{l+s}{\PYZsq{}}\PY{p}{)}
         \PY{n}{legend}\PY{p}{(}\PY{p}{[}\PY{l+s}{\PYZsq{}}\PY{l+s}{original}\PY{l+s}{\PYZsq{}}\PY{p}{,}\PY{l+s}{\PYZsq{}}\PY{l+s}{plus noise}\PY{l+s}{\PYZsq{}}\PY{p}{,} \PY{l+s}{\PYZsq{}}\PY{l+s}{regression}\PY{l+s}{\PYZsq{}}\PY{p}{,} \PY{l+s}{\PYZsq{}}\PY{l+s}{regression2}\PY{l+s}{\PYZsq{}}\PY{p}{,}\PY{l+s}{\PYZsq{}}\PY{l+s}{regression3}\PY{l+s}{\PYZsq{}}\PY{p}{]}\PY{p}{)}
         
         \PY{n}{show}\PY{p}{(}\PY{p}{)}
         
         \PY{c}{\PYZsh{}Linear regression using stats.linregress}
         \PY{p}{(}\PY{n}{a\PYZus{}s}\PY{p}{,}\PY{n}{b\PYZus{}s}\PY{p}{,}\PY{n}{r}\PY{p}{,}\PY{n}{tt}\PY{p}{,}\PY{n}{stderr}\PY{p}{)}\PY{o}{=}\PY{n}{stats}\PY{o}{.}\PY{n}{linregress}\PY{p}{(}\PY{n}{t}\PY{p}{,}\PY{n}{xn}\PY{p}{)}
         \PY{k}{print}\PY{p}{(}\PY{l+s}{\PYZsq{}}\PY{l+s}{Linear regression using stats.linregress}\PY{l+s}{\PYZsq{}}\PY{p}{)}
         \PY{k}{print}\PY{p}{(}\PY{l+s}{\PYZsq{}}\PY{l+s}{parameters: a=}\PY{l+s+si}{\PYZpc{}.2f}\PY{l+s}{ b=}\PY{l+s+si}{\PYZpc{}.2f}\PY{l+s}{ }\PY{l+s+se}{\PYZbs{}n}\PY{l+s}{regression: a=}\PY{l+s+si}{\PYZpc{}.2f}\PY{l+s}{ b=}\PY{l+s+si}{\PYZpc{}.2f}\PY{l+s}{, std error= }\PY{l+s+si}{\PYZpc{}.3f}\PY{l+s}{\PYZsq{}} \PY{o}{\PYZpc{}} \PY{p}{(}\PY{n}{a}\PY{p}{,}\PY{n}{b}\PY{p}{,}\PY{n}{a\PYZus{}s}\PY{p}{,}\PY{n}{b\PYZus{}s}\PY{p}{,}\PY{n}{stderr}\PY{p}{)}\PY{p}{)}
\end{Verbatim}

    \begin{Verbatim}[commandchars=\\\{\}]
0.873819998785 -4.0359601305
Linear regression using polyfit
parameters: a=0.80 b=-4.00 
regression: a=0.87 b=-4.04, ms error= 0.945
    \end{Verbatim}

    \begin{center}
    \adjustimage{max size={0.9\linewidth}{0.9\paperheight}}{Instrumentacion_files/Instrumentacion_7_1.png}
    \end{center}
    { \hspace*{\fill} \\}
    
    \begin{Verbatim}[commandchars=\\\{\}]
Linear regression using stats.linregress
parameters: a=0.80 b=-4.00 
regression: a=0.87 b=-4.04, std error= 0.046
    \end{Verbatim}

    \begin{Verbatim}[commandchars=\\\{\}]
{\color{incolor}In [{\color{incolor}76}]:} \PY{c}{\PYZsh{}matplotlib ploting}
         \PY{n}{matplotlib}\PY{o}{.}\PY{n}{rcParams}\PY{o}{.}\PY{n}{update}\PY{p}{(}\PY{p}{\PYZob{}}\PY{l+s}{\PYZsq{}}\PY{l+s}{font.size}\PY{l+s}{\PYZsq{}}\PY{p}{:} \PY{l+m+mi}{18}\PY{p}{\PYZcb{}}\PY{p}{)}
         \PY{n}{figure}\PY{p}{(}\PY{n}{figsize}\PY{o}{=}\PY{p}{(}\PY{l+m+mi}{14}\PY{p}{,}\PY{l+m+mi}{10}\PY{p}{)}\PY{p}{,} \PY{n}{dpi}\PY{o}{=}\PY{l+m+mi}{150}\PY{p}{)}
         \PY{n}{title}\PY{p}{(}\PY{l+s}{\PYZsq{}}\PY{l+s}{Linear Regression Example}\PY{l+s}{\PYZsq{}}\PY{p}{)}
         \PY{n}{plot}\PY{p}{(}\PY{n}{t}\PY{p}{,}\PY{n}{x}\PY{p}{,}\PY{l+s}{\PYZsq{}}\PY{l+s}{g.\PYZhy{}\PYZhy{}}\PY{l+s}{\PYZsq{}}\PY{p}{)}
         \PY{n}{plot}\PY{p}{(}\PY{n}{t}\PY{p}{,}\PY{n}{xn}\PY{p}{,}\PY{l+s}{\PYZsq{}}\PY{l+s}{k.}\PY{l+s}{\PYZsq{}}\PY{p}{)}
         \PY{n}{plot}\PY{p}{(}\PY{n}{t}\PY{p}{,}\PY{n}{xr}\PY{p}{,}\PY{l+s}{\PYZsq{}}\PY{l+s}{r.\PYZhy{}}\PY{l+s}{\PYZsq{}}\PY{p}{)}
         \PY{n}{plot}\PY{p}{(}\PY{n}{t}\PY{p}{,}\PY{n}{xr2}\PY{p}{,}\PY{l+s}{\PYZsq{}}\PY{l+s}{b\PYZhy{}\PYZhy{}}\PY{l+s}{\PYZsq{}}\PY{p}{)}
         \PY{n}{plot}\PY{p}{(}\PY{n}{t}\PY{p}{,}\PY{n}{xr3}\PY{p}{,}\PY{l+s}{\PYZsq{}}\PY{l+s}{yx}\PY{l+s}{\PYZsq{}}\PY{p}{)}
         \PY{n}{legend}\PY{p}{(}\PY{p}{[}\PY{l+s}{\PYZsq{}}\PY{l+s}{original}\PY{l+s}{\PYZsq{}}\PY{p}{,}\PY{l+s}{\PYZsq{}}\PY{l+s}{plus noise}\PY{l+s}{\PYZsq{}}\PY{p}{,} \PY{l+s}{\PYZsq{}}\PY{l+s}{regression}\PY{l+s}{\PYZsq{}}\PY{p}{,} \PY{l+s}{\PYZsq{}}\PY{l+s}{regression2}\PY{l+s}{\PYZsq{}}\PY{p}{,}\PY{l+s}{\PYZsq{}}\PY{l+s}{regression3}\PY{l+s}{\PYZsq{}}\PY{p}{]}\PY{p}{)}
         \PY{n}{xlim}\PY{p}{(}\PY{n}{xmin}\PY{o}{=}\PY{o}{\PYZhy{}}\PY{l+m+mf}{5.1}\PY{p}{,} \PY{n}{xmax}\PY{o}{=}\PY{o}{\PYZhy{}}\PY{l+m+mi}{2}\PY{p}{)}
         \PY{n}{ylim}\PY{p}{(}\PY{n}{ymin}\PY{o}{=}\PY{o}{\PYZhy{}}\PY{l+m+mi}{8}\PY{p}{,} \PY{n}{ymax}\PY{o}{=}\PY{o}{\PYZhy{}}\PY{l+m+mi}{6}\PY{p}{)}
         \PY{n}{show}\PY{p}{(}\PY{p}{)}
\end{Verbatim}

    \begin{center}
    \adjustimage{max size={0.9\linewidth}{0.9\paperheight}}{Instrumentacion_files/Instrumentacion_8_0.png}
    \end{center}
    { \hspace*{\fill} \\}
    
    \subsection{Sistema de Medición por divisor de
voltaje}\label{sistema-de-mediciuxf3n-por-divisor-de-voltaje}

Este es un ejemplo de como utilizar un simple divisor como sistema para
medir la luz, se tiene qun divisor de voltaje tal que el voltaje de
salida es

\begin{equation}
V_{o} = \frac{R_{2}}{R_{1}+R_{2}}V_{i}
\end{equation}

Si consideramos que $R_{2}$ es un resistencia que varia con respecto a
la intencidad de luz ($L$) (0\% a 100\%), de forma tal que:

\begin{equation}
R_{2}=kR_{b}
\end{equation}

donde $k$ es una ganancia y $R_{b}$ es la resistencia base. Entonces es
posible realiconar el voltaje de salida $V_{o}$ con la intencidad de luz
$L$. Qeudando el voltaje de salida como:

\begin{equation}
V_{o} = \frac{R_{b}k}{R_{1}+R_{b}k}V_{i}
\end{equation}

    \section{Uso de algebra
computacional}\label{uso-de-algebra-computacional}

Para facilitar las operaciones algebraicas en la actualidad existen
herramientas para simplificar los calculos algebraicos, tal como sympy
una libreria de python que permite el Computo de Algebre Simbolica

    \begin{Verbatim}[commandchars=\\\{\}]
{\color{incolor}In [{\color{incolor}92}]:} \PY{k+kn}{import} \PY{n+nn}{sympy}
\end{Verbatim}

    \begin{Verbatim}[commandchars=\\\{\}]
{\color{incolor}In [{\color{incolor}93}]:} \PY{n}{R2}\PY{p}{,} \PY{n}{R1}\PY{p}{,} \PY{n}{Vi}\PY{p}{,} \PY{n}{Vo} \PY{o}{=} \PY{n}{sympy}\PY{o}{.}\PY{n}{symbols}\PY{p}{(}\PY{l+s}{\PYZsq{}}\PY{l+s}{R2 R1 Vi Vo}\PY{l+s}{\PYZsq{}}\PY{p}{)}
         \PY{n}{Vo}\PY{o}{=}\PY{p}{(}\PY{p}{(}\PY{n}{R2}\PY{p}{)}\PY{o}{/}\PY{p}{(}\PY{n}{R2}\PY{o}{+}\PY{n}{R1}\PY{p}{)}\PY{p}{)}\PY{o}{*}\PY{n}{Vi}
         \PY{k}{print}\PY{p}{(}\PY{n}{Vo}\PY{p}{)}
\end{Verbatim}

    \begin{Verbatim}[commandchars=\\\{\}]
R2*Vi/(R1 + R2)
    \end{Verbatim}

    \begin{Verbatim}[commandchars=\\\{\}]
{\color{incolor}In [{\color{incolor}94}]:} \PY{n}{k}\PY{p}{,} \PY{n}{Rb}\PY{p}{,} \PY{n}{R1}\PY{p}{,} \PY{n}{Vi}\PY{p}{,} \PY{n}{Vo} \PY{o}{=} \PY{n}{sympy}\PY{o}{.}\PY{n}{symbols}\PY{p}{(}\PY{l+s}{\PYZsq{}}\PY{l+s}{k Rb R1 Vi Vo}\PY{l+s}{\PYZsq{}}\PY{p}{)}
         \PY{n}{R2}\PY{o}{=}\PY{n}{k}\PY{o}{*}\PY{n}{Rb}
         \PY{n}{Vo}\PY{o}{=}\PY{p}{(}\PY{n}{R2}\PY{o}{*}\PY{n}{Vi}\PY{p}{)}\PY{o}{/}\PY{p}{(}\PY{n}{R2}\PY{o}{+}\PY{n}{R1}\PY{p}{)}
         \PY{k}{print}\PY{p}{(}\PY{n}{Vo}\PY{p}{)}
\end{Verbatim}

    \begin{Verbatim}[commandchars=\\\{\}]
Rb*Vi*k/(R1 + Rb*k)
    \end{Verbatim}

    \begin{Verbatim}[commandchars=\\\{\}]
{\color{incolor}In [{\color{incolor}95}]:} \PY{n}{sympy}\PY{o}{.}\PY{n}{simplify}\PY{p}{(}\PY{n}{Vo}\PY{p}{)}
\end{Verbatim}

            \begin{Verbatim}[commandchars=\\\{\}]
{\color{outcolor}Out[{\color{outcolor}95}]:} Rb*Vi*k/(R1 + Rb*k)
\end{Verbatim}
        
    \begin{Verbatim}[commandchars=\\\{\}]
{\color{incolor}In [{\color{incolor}96}]:} \PY{n}{sympy}\PY{o}{.}\PY{n}{factor}\PY{p}{(}\PY{n}{Vo}\PY{p}{)}
\end{Verbatim}

            \begin{Verbatim}[commandchars=\\\{\}]
{\color{outcolor}Out[{\color{outcolor}96}]:} Rb*Vi*k/(R1 + Rb*k)
\end{Verbatim}
        
    Es importante señalar que la simplificación computacional no siempre
brinda los resultados optimos, por lo que es necesario tener cuidado con
su uso, en este caso un expresión más simple podría ser:

\begin{equation}
V_{o} = \frac{R_{b}}{\frac{R_{1}}{k}+R_{b}}V_{i}
\end{equation}

    \begin{Verbatim}[commandchars=\\\{\}]
{\color{incolor}In [{\color{incolor}97}]:} \PY{n}{sympy}\PY{o}{.}\PY{n}{simplify}\PY{p}{(}\PY{p}{(}\PY{n}{Vi}\PY{o}{*}\PY{n}{Rb}\PY{p}{)}\PY{o}{/}\PY{p}{(}\PY{p}{(}\PY{n}{R1}\PY{o}{/}\PY{n}{k}\PY{p}{)}\PY{o}{+}\PY{n}{Rb}\PY{p}{)}\PY{p}{)}
\end{Verbatim}

            \begin{Verbatim}[commandchars=\\\{\}]
{\color{outcolor}Out[{\color{outcolor}97}]:} Rb*Vi*k/(R1 + Rb*k)
\end{Verbatim}
        

    \subsection{Cálculo Númerico}


    Realizando un cálculo númerico suponiendo que: $R_{1}=2.5k\Omega$,
$V_{i}=10$ y que $R_{2}$ tiene un valor entre $(1k\Omega)$ a
$(5k\Omega)$ que corresponden a $L$ entre $0\%$ y $100\%$

    \begin{Verbatim}[commandchars=\\\{\}]
{\color{incolor}In [{\color{incolor}98}]:} \PY{k+kn}{import} \PY{n+nn}{numpy} \PY{k+kn}{as} \PY{n+nn}{np}
\end{Verbatim}

    \begin{Verbatim}[commandchars=\\\{\}]
{\color{incolor}In [{\color{incolor}99}]:} \PY{n}{R2}\PY{o}{=}\PY{n}{linspace}\PY{p}{(}\PY{l+m+mi}{100}\PY{p}{,}\PY{l+m+mi}{5000}\PY{p}{,}\PY{l+m+mi}{1000}\PY{p}{)}
         \PY{n}{R1}\PY{o}{=}\PY{l+m+mi}{2500}
         \PY{n}{Vi}\PY{o}{=}\PY{l+m+mi}{10}
         \PY{n}{L}\PY{o}{=}\PY{n}{linspace}\PY{p}{(}\PY{l+m+mi}{0}\PY{p}{,}\PY{l+m+mi}{100}\PY{p}{,}\PY{l+m+mi}{1000}\PY{p}{)}
\end{Verbatim}

    \begin{Verbatim}[commandchars=\\\{\}]
{\color{incolor}In [{\color{incolor}100}]:} \PY{n}{Vo}\PY{o}{=}\PY{p}{(}\PY{n}{R2}\PY{o}{*}\PY{n}{Vi}\PY{p}{)}\PY{o}{/}\PY{p}{(}\PY{n}{R1}\PY{o}{+}\PY{n}{R2}\PY{p}{)}
\end{Verbatim}

    \begin{Verbatim}[commandchars=\\\{\}]
{\color{incolor}In [{\color{incolor}101}]:} \PY{n}{matplotlib}\PY{o}{.}\PY{n}{rcParams}\PY{o}{.}\PY{n}{update}\PY{p}{(}\PY{p}{\PYZob{}}\PY{l+s}{\PYZsq{}}\PY{l+s}{font.size}\PY{l+s}{\PYZsq{}}\PY{p}{:} \PY{l+m+mi}{18}\PY{p}{\PYZcb{}}\PY{p}{)}
          \PY{n}{figure}\PY{p}{(}\PY{n}{figsize}\PY{o}{=}\PY{p}{(}\PY{l+m+mi}{14}\PY{p}{,}\PY{l+m+mi}{10}\PY{p}{)}\PY{p}{,} \PY{n}{dpi}\PY{o}{=}\PY{l+m+mi}{150}\PY{p}{)}
          \PY{n}{xlabel}\PY{p}{(}\PY{l+s}{\PYZsq{}}\PY{l+s}{R2}\PY{l+s}{\PYZsq{}}\PY{p}{,} \PY{n}{fontsize}\PY{o}{=}\PY{l+m+mi}{20}\PY{p}{)}
          \PY{n}{ylabel}\PY{p}{(}\PY{l+s}{\PYZsq{}}\PY{l+s}{Vo}\PY{l+s}{\PYZsq{}}\PY{p}{,} \PY{n}{fontsize}\PY{o}{=}\PY{l+m+mi}{20}\PY{p}{)}
          \PY{n}{xlim}\PY{p}{(}\PY{n}{xmin}\PY{o}{=}\PY{l+m+mi}{100}\PY{p}{,} \PY{n}{xmax}\PY{o}{=}\PY{l+m+mi}{5000}\PY{p}{)}
          \PY{n}{ylim}\PY{p}{(}\PY{n}{ymin}\PY{o}{=}\PY{l+m+mf}{0.5}\PY{p}{,} \PY{n}{ymax}\PY{o}{=}\PY{l+m+mi}{7}\PY{p}{)}
          \PY{n}{plot}\PY{p}{(}\PY{n}{R2}\PY{p}{,}\PY{n}{Vo}\PY{p}{,} \PY{n}{color}\PY{o}{=}\PY{l+s}{\PYZdq{}}\PY{l+s}{blue}\PY{l+s}{\PYZdq{}}\PY{p}{,} \PY{n}{linewidth}\PY{o}{=}\PY{l+m+mf}{2.5}\PY{p}{,} \PY{n}{linestyle}\PY{o}{=}\PY{l+s}{\PYZdq{}}\PY{l+s}{\PYZhy{}}\PY{l+s}{\PYZdq{}}\PY{p}{)}
          \PY{n}{grid}\PY{p}{(}\PY{p}{)}
\end{Verbatim}

    \begin{center}
    \adjustimage{max size={0.9\linewidth}{0.9\paperheight}}{Instrumentacion_files/Instrumentacion_23_0.png}
    \end{center}
    { \hspace*{\fill} \\}
    
    \begin{Verbatim}[commandchars=\\\{\}]
{\color{incolor}In [{\color{incolor}102}]:} \PY{n}{figure}\PY{p}{(}\PY{n}{figsize}\PY{o}{=}\PY{p}{(}\PY{l+m+mi}{14}\PY{p}{,}\PY{l+m+mi}{10}\PY{p}{)}\PY{p}{,} \PY{n}{dpi}\PY{o}{=}\PY{l+m+mi}{150}\PY{p}{)}
          \PY{n}{xlabel}\PY{p}{(}\PY{l+s}{\PYZsq{}}\PY{l+s}{Vo}\PY{l+s}{\PYZsq{}}\PY{p}{,} \PY{n}{fontsize}\PY{o}{=}\PY{l+m+mi}{20}\PY{p}{)}
          \PY{n}{ylabel}\PY{p}{(}\PY{l+s}{\PYZsq{}}\PY{l+s}{L}\PY{l+s}{\PYZsq{}}\PY{p}{,} \PY{n}{fontsize}\PY{o}{=}\PY{l+m+mi}{20}\PY{p}{)}
          \PY{n}{xlim}\PY{p}{(}\PY{n}{xmin}\PY{o}{=}\PY{l+m+mi}{0}\PY{p}{,} \PY{n}{xmax}\PY{o}{=}\PY{l+m+mi}{7}\PY{p}{)}
          \PY{n}{ylim}\PY{p}{(}\PY{n}{ymin}\PY{o}{=}\PY{o}{\PYZhy{}}\PY{l+m+mf}{0.5}\PY{p}{,} \PY{n}{ymax}\PY{o}{=}\PY{l+m+mi}{105}\PY{p}{)}
          \PY{n}{plot}\PY{p}{(}\PY{n}{Vo}\PY{p}{,}\PY{n}{L}\PY{p}{,} \PY{n}{color}\PY{o}{=}\PY{l+s}{\PYZdq{}}\PY{l+s}{blue}\PY{l+s}{\PYZdq{}}\PY{p}{,} \PY{n}{linewidth}\PY{o}{=}\PY{l+m+mf}{2.5}\PY{p}{,} \PY{n}{linestyle}\PY{o}{=}\PY{l+s}{\PYZdq{}}\PY{l+s}{\PYZhy{}}\PY{l+s}{\PYZdq{}}\PY{p}{)}
          \PY{n}{grid}\PY{p}{(}\PY{p}{)}
\end{Verbatim}

    \begin{center}
    \adjustimage{max size={0.9\linewidth}{0.9\paperheight}}{Instrumentacion_files/Instrumentacion_24_0.png}
    \end{center}
    { \hspace*{\fill} \\}
    
    \begin{Verbatim}[commandchars=\\\{\}]
{\color{incolor}In [{\color{incolor}103}]:} \PY{n}{s}\PY{o}{=}\PY{n}{Vo}\PY{o}{.}\PY{n}{size}
          \PY{k}{print}\PY{p}{(}\PY{n}{s}\PY{o}{/}\PY{l+m+mi}{100}\PY{p}{)}
          \PY{n}{ini}\PY{o}{=}\PY{n}{np}\PY{o}{.}\PY{n}{min}\PY{p}{(}\PY{n}{Vo}\PY{p}{)}
          \PY{n}{fin}\PY{o}{=}\PY{n}{np}\PY{o}{.}\PY{n}{max}\PY{p}{(}\PY{n}{Vo}\PY{p}{)}
          \PY{n}{Vop}\PY{o}{=}\PY{n}{np}\PY{o}{.}\PY{n}{zeros}\PY{p}{(}\PY{p}{[}\PY{n}{s}\PY{o}{/}\PY{l+m+mi}{100}\PY{p}{]}\PY{p}{)}
          \PY{n}{Lp}\PY{o}{=}\PY{n}{np}\PY{o}{.}\PY{n}{zeros}\PY{p}{(}\PY{p}{[}\PY{n}{s}\PY{o}{/}\PY{l+m+mi}{100}\PY{p}{]}\PY{p}{)}
          \PY{n}{k}\PY{o}{=}\PY{l+m+mi}{0}
          \PY{k}{for} \PY{n}{i} \PY{o+ow}{in} \PY{n+nb}{range}\PY{p}{(}\PY{l+m+mi}{0}\PY{p}{,}\PY{n+nb}{int}\PY{p}{(}\PY{n}{s}\PY{p}{)}\PY{p}{,}\PY{n+nb}{int}\PY{p}{(}\PY{n}{s}\PY{o}{/}\PY{l+m+mi}{10}\PY{p}{)}\PY{p}{)}\PY{p}{:}
              \PY{n}{Vop}\PY{p}{[}\PY{n}{k}\PY{p}{]}\PY{o}{=}\PY{n}{Vo}\PY{p}{[}\PY{n}{i}\PY{p}{]}
              \PY{n}{Lp}\PY{p}{[}\PY{n}{k}\PY{p}{]}\PY{o}{=}\PY{n}{L}\PY{p}{[}\PY{n}{i}\PY{p}{]}
              \PY{n}{k} \PY{o}{+}\PY{o}{=} \PY{l+m+mi}{1}
\end{Verbatim}

    \begin{Verbatim}[commandchars=\\\{\}]
10.0
    \end{Verbatim}

    \begin{Verbatim}[commandchars=\\\{\}]
{\color{incolor}In [{\color{incolor}104}]:} \PY{n}{figure}\PY{p}{(}\PY{n}{figsize}\PY{o}{=}\PY{p}{(}\PY{l+m+mi}{14}\PY{p}{,}\PY{l+m+mi}{10}\PY{p}{)}\PY{p}{,} \PY{n}{dpi}\PY{o}{=}\PY{l+m+mi}{150}\PY{p}{)}
          \PY{n}{z}\PY{o}{=}\PY{n}{np}\PY{o}{.}\PY{n}{polyfit}\PY{p}{(}\PY{n}{Vop}\PY{p}{,} \PY{n}{Lp}\PY{p}{,} \PY{l+m+mi}{2}\PY{p}{)}
          \PY{k}{print}\PY{p}{(}\PY{n}{z}\PY{p}{)}
          \PY{n}{p} \PY{o}{=} \PY{n}{np}\PY{o}{.}\PY{n}{poly1d}\PY{p}{(}\PY{n}{z}\PY{p}{)}
          \PY{k}{print}\PY{p}{(}\PY{n}{p}\PY{p}{)}
          \PY{n}{figure}\PY{p}{(}\PY{n}{figsize}\PY{o}{=}\PY{p}{(}\PY{l+m+mi}{14}\PY{p}{,}\PY{l+m+mi}{10}\PY{p}{)}\PY{p}{,} \PY{n}{dpi}\PY{o}{=}\PY{l+m+mi}{150}\PY{p}{)}
          \PY{n}{xlabel}\PY{p}{(}\PY{l+s}{\PYZsq{}}\PY{l+s}{Vo}\PY{l+s}{\PYZsq{}}\PY{p}{,} \PY{n}{fontsize}\PY{o}{=}\PY{l+m+mi}{20}\PY{p}{)}
          \PY{n}{ylabel}\PY{p}{(}\PY{l+s}{\PYZsq{}}\PY{l+s}{L}\PY{l+s}{\PYZsq{}}\PY{p}{,} \PY{n}{fontsize}\PY{o}{=}\PY{l+m+mi}{20}\PY{p}{)}
          \PY{n}{xlim}\PY{p}{(}\PY{n}{xmin}\PY{o}{=}\PY{l+m+mi}{0}\PY{p}{,} \PY{n}{xmax}\PY{o}{=}\PY{l+m+mi}{7}\PY{p}{)}
          \PY{n}{ylim}\PY{p}{(}\PY{n}{ymin}\PY{o}{=}\PY{o}{\PYZhy{}}\PY{l+m+mf}{0.5}\PY{p}{,} \PY{n}{ymax}\PY{o}{=}\PY{l+m+mi}{105}\PY{p}{)}
          \PY{n}{plot}\PY{p}{(}\PY{n}{Vop}\PY{p}{,}\PY{n}{Lp}\PY{p}{,}\PY{l+s}{\PYZsq{}}\PY{l+s}{.}\PY{l+s}{\PYZsq{}}\PY{p}{,} \PY{n}{color}\PY{o}{=}\PY{l+s}{\PYZdq{}}\PY{l+s}{blue}\PY{l+s}{\PYZdq{}}\PY{p}{,}\PY{n}{markersize}\PY{o}{=}\PY{l+m+mi}{15}\PY{p}{)}
          \PY{n}{plot}\PY{p}{(}\PY{n}{Vo}\PY{p}{,}\PY{n}{L}\PY{p}{,} \PY{n}{color}\PY{o}{=}\PY{l+s}{\PYZdq{}}\PY{l+s}{Red}\PY{l+s}{\PYZdq{}}\PY{p}{,} \PY{n}{linewidth}\PY{o}{=}\PY{l+m+mf}{2.5}\PY{p}{,} \PY{n}{linestyle}\PY{o}{=}\PY{l+s}{\PYZdq{}}\PY{l+s}{\PYZhy{}}\PY{l+s}{\PYZdq{}}\PY{p}{)}
          \PY{n}{plot}\PY{p}{(}\PY{n}{Vo}\PY{p}{,}\PY{n}{p}\PY{p}{(}\PY{n}{Vo}\PY{p}{)}\PY{p}{,} \PY{n}{color}\PY{o}{=}\PY{l+s}{\PYZdq{}}\PY{l+s}{green}\PY{l+s}{\PYZdq{}}\PY{p}{,} \PY{n}{linewidth}\PY{o}{=}\PY{l+m+mf}{2.5}\PY{p}{,} \PY{n}{linestyle}\PY{o}{=}\PY{l+s}{\PYZdq{}}\PY{l+s}{\PYZhy{}}\PY{l+s}{\PYZdq{}}\PY{p}{)}
          \PY{n}{grid}\PY{p}{(}\PY{p}{)}
          \PY{n}{figure}\PY{p}{(}\PY{n}{figsize}\PY{o}{=}\PY{p}{(}\PY{l+m+mi}{14}\PY{p}{,}\PY{l+m+mi}{10}\PY{p}{)}\PY{p}{,} \PY{n}{dpi}\PY{o}{=}\PY{l+m+mi}{150}\PY{p}{)}
          \PY{n}{xlabel}\PY{p}{(}\PY{l+s}{\PYZsq{}}\PY{l+s}{L}\PY{l+s}{\PYZsq{}}\PY{p}{,} \PY{n}{fontsize}\PY{o}{=}\PY{l+m+mi}{20}\PY{p}{)}
          \PY{n}{ylabel}\PY{p}{(}\PY{l+s}{\PYZsq{}}\PY{l+s}{Error}\PY{l+s}{\PYZsq{}}\PY{p}{,} \PY{n}{fontsize}\PY{o}{=}\PY{l+m+mi}{20}\PY{p}{)}
          \PY{n}{plot}\PY{p}{(}\PY{n}{L}\PY{p}{,}\PY{p}{(}\PY{n}{p}\PY{p}{(}\PY{n}{Vo}\PY{p}{)}\PY{o}{\PYZhy{}}\PY{n}{L}\PY{p}{)}\PY{p}{,} \PY{n}{color}\PY{o}{=}\PY{l+s}{\PYZdq{}}\PY{l+s}{green}\PY{l+s}{\PYZdq{}}\PY{p}{,} \PY{n}{linewidth}\PY{o}{=}\PY{l+m+mf}{2.5}\PY{p}{,} \PY{n}{linestyle}\PY{o}{=}\PY{l+s}{\PYZdq{}}\PY{l+s}{\PYZhy{}}\PY{l+s}{\PYZdq{}}\PY{p}{)}
\end{Verbatim}

    \begin{Verbatim}[commandchars=\\\{\}]
[ 2.32428664 -1.83149147  2.27442022]
       2
2.324 x - 1.831 x + 2.274
    \end{Verbatim}

            \begin{Verbatim}[commandchars=\\\{\}]
{\color{outcolor}Out[{\color{outcolor}104}]:} [<matplotlib.lines.Line2D at 0x110c1ac50>]
\end{Verbatim}
        
    
    \begin{verbatim}
<matplotlib.figure.Figure at 0x110e65d10>
    \end{verbatim}

    
    \begin{center}
    \adjustimage{max size={0.9\linewidth}{0.9\paperheight}}{Instrumentacion_files/Instrumentacion_26_3.png}
    \end{center}
    { \hspace*{\fill} \\}
    
    \begin{center}
    \adjustimage{max size={0.9\linewidth}{0.9\paperheight}}{Instrumentacion_files/Instrumentacion_26_4.png}
    \end{center}
    { \hspace*{\fill} \\}
    
    \begin{Verbatim}[commandchars=\\\{\}]
{\color{incolor}In [{\color{incolor}80}]:} \PY{n}{figure}\PY{p}{(}\PY{n}{figsize}\PY{o}{=}\PY{p}{(}\PY{l+m+mi}{14}\PY{p}{,}\PY{l+m+mi}{10}\PY{p}{)}\PY{p}{,} \PY{n}{dpi}\PY{o}{=}\PY{l+m+mi}{150}\PY{p}{)}
         \PY{n}{z}\PY{o}{=}\PY{n}{np}\PY{o}{.}\PY{n}{polyfit}\PY{p}{(}\PY{n}{Vop}\PY{p}{,} \PY{n}{Lp}\PY{p}{,} \PY{l+m+mi}{3}\PY{p}{)}
         \PY{k}{print}\PY{p}{(}\PY{n}{z}\PY{p}{)}
         \PY{n}{p} \PY{o}{=} \PY{n}{np}\PY{o}{.}\PY{n}{poly1d}\PY{p}{(}\PY{n}{z}\PY{p}{)}
         \PY{k}{print}\PY{p}{(}\PY{n}{p}\PY{p}{)}
         \PY{n}{figure}\PY{p}{(}\PY{n}{figsize}\PY{o}{=}\PY{p}{(}\PY{l+m+mi}{14}\PY{p}{,}\PY{l+m+mi}{10}\PY{p}{)}\PY{p}{,} \PY{n}{dpi}\PY{o}{=}\PY{l+m+mi}{150}\PY{p}{)}
         \PY{n}{xlabel}\PY{p}{(}\PY{l+s}{\PYZsq{}}\PY{l+s}{Vo}\PY{l+s}{\PYZsq{}}\PY{p}{,} \PY{n}{fontsize}\PY{o}{=}\PY{l+m+mi}{20}\PY{p}{)}
         \PY{n}{ylabel}\PY{p}{(}\PY{l+s}{\PYZsq{}}\PY{l+s}{L}\PY{l+s}{\PYZsq{}}\PY{p}{,} \PY{n}{fontsize}\PY{o}{=}\PY{l+m+mi}{20}\PY{p}{)}
         \PY{n}{xlim}\PY{p}{(}\PY{n}{xmin}\PY{o}{=}\PY{l+m+mi}{0}\PY{p}{,} \PY{n}{xmax}\PY{o}{=}\PY{l+m+mi}{7}\PY{p}{)}
         \PY{n}{ylim}\PY{p}{(}\PY{n}{ymin}\PY{o}{=}\PY{o}{\PYZhy{}}\PY{l+m+mf}{0.5}\PY{p}{,} \PY{n}{ymax}\PY{o}{=}\PY{l+m+mi}{105}\PY{p}{)}
         \PY{n}{plot}\PY{p}{(}\PY{n}{Vop}\PY{p}{,}\PY{n}{Lp}\PY{p}{,}\PY{l+s}{\PYZsq{}}\PY{l+s}{.}\PY{l+s}{\PYZsq{}}\PY{p}{,} \PY{n}{color}\PY{o}{=}\PY{l+s}{\PYZdq{}}\PY{l+s}{blue}\PY{l+s}{\PYZdq{}}\PY{p}{,}\PY{n}{markersize}\PY{o}{=}\PY{l+m+mi}{15}\PY{p}{)}
         \PY{n}{plot}\PY{p}{(}\PY{n}{Vo}\PY{p}{,}\PY{n}{L}\PY{p}{,} \PY{n}{color}\PY{o}{=}\PY{l+s}{\PYZdq{}}\PY{l+s}{Red}\PY{l+s}{\PYZdq{}}\PY{p}{,} \PY{n}{linewidth}\PY{o}{=}\PY{l+m+mf}{2.5}\PY{p}{,} \PY{n}{linestyle}\PY{o}{=}\PY{l+s}{\PYZdq{}}\PY{l+s}{\PYZhy{}}\PY{l+s}{\PYZdq{}}\PY{p}{)}
         \PY{n}{plot}\PY{p}{(}\PY{n}{Vo}\PY{p}{,}\PY{n}{p}\PY{p}{(}\PY{n}{Vo}\PY{p}{)}\PY{p}{,} \PY{n}{color}\PY{o}{=}\PY{l+s}{\PYZdq{}}\PY{l+s}{green}\PY{l+s}{\PYZdq{}}\PY{p}{,} \PY{n}{linewidth}\PY{o}{=}\PY{l+m+mf}{2.5}\PY{p}{,} \PY{n}{linestyle}\PY{o}{=}\PY{l+s}{\PYZdq{}}\PY{l+s}{\PYZhy{}}\PY{l+s}{\PYZdq{}}\PY{p}{)}
         \PY{n}{grid}\PY{p}{(}\PY{p}{)}
         \PY{n}{figure}\PY{p}{(}\PY{n}{figsize}\PY{o}{=}\PY{p}{(}\PY{l+m+mi}{14}\PY{p}{,}\PY{l+m+mi}{10}\PY{p}{)}\PY{p}{,} \PY{n}{dpi}\PY{o}{=}\PY{l+m+mi}{150}\PY{p}{)}
         \PY{n}{xlabel}\PY{p}{(}\PY{l+s}{\PYZsq{}}\PY{l+s}{L}\PY{l+s}{\PYZsq{}}\PY{p}{,} \PY{n}{fontsize}\PY{o}{=}\PY{l+m+mi}{20}\PY{p}{)}
         \PY{n}{ylabel}\PY{p}{(}\PY{l+s}{\PYZsq{}}\PY{l+s}{Error}\PY{l+s}{\PYZsq{}}\PY{p}{,} \PY{n}{fontsize}\PY{o}{=}\PY{l+m+mi}{20}\PY{p}{)}
         \PY{n}{plot}\PY{p}{(}\PY{n}{L}\PY{p}{,}\PY{p}{(}\PY{n}{p}\PY{p}{(}\PY{n}{Vo}\PY{p}{)}\PY{o}{\PYZhy{}}\PY{n}{L}\PY{p}{)}\PY{p}{,} \PY{n}{color}\PY{o}{=}\PY{l+s}{\PYZdq{}}\PY{l+s}{green}\PY{l+s}{\PYZdq{}}\PY{p}{,} \PY{n}{linewidth}\PY{o}{=}\PY{l+m+mf}{2.5}\PY{p}{,} \PY{n}{linestyle}\PY{o}{=}\PY{l+s}{\PYZdq{}}\PY{l+s}{\PYZhy{}}\PY{l+s}{\PYZdq{}}\PY{p}{)}
\end{Verbatim}

    \begin{Verbatim}[commandchars=\\\{\}]

        ---------------------------------------------------------------------------
    NameError                                 Traceback (most recent call last)

        <ipython-input-80-3287b09e098d> in <module>()
          1 figure(figsize=(14,10), dpi=150)
    ----> 2 z=np.polyfit(Vop, Lp, 3)
          3 print(z)
          4 p = np.poly1d(z)
          5 print(p)


        NameError: name 'Vop' is not defined

    \end{Verbatim}

    
    \begin{verbatim}
<matplotlib.figure.Figure at 0x10e66a9d0>
    \end{verbatim}

    
    \begin{Verbatim}[commandchars=\\\{\}]
{\color{incolor}In [{\color{incolor}105}]:} \PY{n}{figure}\PY{p}{(}\PY{n}{figsize}\PY{o}{=}\PY{p}{(}\PY{l+m+mi}{14}\PY{p}{,}\PY{l+m+mi}{10}\PY{p}{)}\PY{p}{,} \PY{n}{dpi}\PY{o}{=}\PY{l+m+mi}{150}\PY{p}{)}
          \PY{n}{z}\PY{o}{=}\PY{n}{np}\PY{o}{.}\PY{n}{polyfit}\PY{p}{(}\PY{n}{Vop}\PY{p}{,} \PY{n}{Lp}\PY{p}{,} \PY{l+m+mi}{4}\PY{p}{)}
          \PY{k}{print}\PY{p}{(}\PY{n}{z}\PY{p}{)}
          \PY{n}{p} \PY{o}{=} \PY{n}{np}\PY{o}{.}\PY{n}{poly1d}\PY{p}{(}\PY{n}{z}\PY{p}{)}
          \PY{k}{print}\PY{p}{(}\PY{n}{p}\PY{p}{)}
          \PY{n}{figure}\PY{p}{(}\PY{n}{figsize}\PY{o}{=}\PY{p}{(}\PY{l+m+mi}{14}\PY{p}{,}\PY{l+m+mi}{10}\PY{p}{)}\PY{p}{,} \PY{n}{dpi}\PY{o}{=}\PY{l+m+mi}{150}\PY{p}{)}
          \PY{n}{xlabel}\PY{p}{(}\PY{l+s}{\PYZsq{}}\PY{l+s}{Vo}\PY{l+s}{\PYZsq{}}\PY{p}{,} \PY{n}{fontsize}\PY{o}{=}\PY{l+m+mi}{20}\PY{p}{)}
          \PY{n}{ylabel}\PY{p}{(}\PY{l+s}{\PYZsq{}}\PY{l+s}{L}\PY{l+s}{\PYZsq{}}\PY{p}{,} \PY{n}{fontsize}\PY{o}{=}\PY{l+m+mi}{20}\PY{p}{)}
          \PY{n}{xlim}\PY{p}{(}\PY{n}{xmin}\PY{o}{=}\PY{l+m+mi}{0}\PY{p}{,} \PY{n}{xmax}\PY{o}{=}\PY{l+m+mi}{7}\PY{p}{)}
          \PY{n}{ylim}\PY{p}{(}\PY{n}{ymin}\PY{o}{=}\PY{o}{\PYZhy{}}\PY{l+m+mf}{0.5}\PY{p}{,} \PY{n}{ymax}\PY{o}{=}\PY{l+m+mi}{105}\PY{p}{)}
          \PY{n}{plot}\PY{p}{(}\PY{n}{Vop}\PY{p}{,}\PY{n}{Lp}\PY{p}{,}\PY{l+s}{\PYZsq{}}\PY{l+s}{.}\PY{l+s}{\PYZsq{}}\PY{p}{,} \PY{n}{color}\PY{o}{=}\PY{l+s}{\PYZdq{}}\PY{l+s}{blue}\PY{l+s}{\PYZdq{}}\PY{p}{,}\PY{n}{markersize}\PY{o}{=}\PY{l+m+mi}{15}\PY{p}{)}
          \PY{n}{plot}\PY{p}{(}\PY{n}{Vo}\PY{p}{,}\PY{n}{L}\PY{p}{,} \PY{n}{color}\PY{o}{=}\PY{l+s}{\PYZdq{}}\PY{l+s}{Red}\PY{l+s}{\PYZdq{}}\PY{p}{,} \PY{n}{linewidth}\PY{o}{=}\PY{l+m+mf}{2.5}\PY{p}{,} \PY{n}{linestyle}\PY{o}{=}\PY{l+s}{\PYZdq{}}\PY{l+s}{\PYZhy{}}\PY{l+s}{\PYZdq{}}\PY{p}{)}
          \PY{n}{plot}\PY{p}{(}\PY{n}{Vo}\PY{p}{,}\PY{n}{p}\PY{p}{(}\PY{n}{Vo}\PY{p}{)}\PY{p}{,} \PY{n}{color}\PY{o}{=}\PY{l+s}{\PYZdq{}}\PY{l+s}{green}\PY{l+s}{\PYZdq{}}\PY{p}{,} \PY{n}{linewidth}\PY{o}{=}\PY{l+m+mf}{2.5}\PY{p}{,} \PY{n}{linestyle}\PY{o}{=}\PY{l+s}{\PYZdq{}}\PY{l+s}{\PYZhy{}}\PY{l+s}{\PYZdq{}}\PY{p}{)}
          \PY{n}{grid}\PY{p}{(}\PY{p}{)}
          \PY{n}{figure}\PY{p}{(}\PY{n}{figsize}\PY{o}{=}\PY{p}{(}\PY{l+m+mi}{14}\PY{p}{,}\PY{l+m+mi}{10}\PY{p}{)}\PY{p}{,} \PY{n}{dpi}\PY{o}{=}\PY{l+m+mi}{150}\PY{p}{)}
          \PY{n}{xlabel}\PY{p}{(}\PY{l+s}{\PYZsq{}}\PY{l+s}{L}\PY{l+s}{\PYZsq{}}\PY{p}{,} \PY{n}{fontsize}\PY{o}{=}\PY{l+m+mi}{20}\PY{p}{)}
          \PY{n}{ylabel}\PY{p}{(}\PY{l+s}{\PYZsq{}}\PY{l+s}{Error}\PY{l+s}{\PYZsq{}}\PY{p}{,} \PY{n}{fontsize}\PY{o}{=}\PY{l+m+mi}{20}\PY{p}{)}
          \PY{n}{plot}\PY{p}{(}\PY{n}{L}\PY{p}{,}\PY{p}{(}\PY{n}{p}\PY{p}{(}\PY{n}{Vo}\PY{p}{)}\PY{o}{\PYZhy{}}\PY{n}{L}\PY{p}{)}\PY{p}{,} \PY{n}{color}\PY{o}{=}\PY{l+s}{\PYZdq{}}\PY{l+s}{green}\PY{l+s}{\PYZdq{}}\PY{p}{,} \PY{n}{linewidth}\PY{o}{=}\PY{l+m+mf}{2.5}\PY{p}{,} \PY{n}{linestyle}\PY{o}{=}\PY{l+s}{\PYZdq{}}\PY{l+s}{\PYZhy{}}\PY{l+s}{\PYZdq{}}\PY{p}{)}
\end{Verbatim}

    \begin{Verbatim}[commandchars=\\\{\}]
[ 0.06635417 -0.54567593  2.5853706   2.45149595 -1.28208164]
         4          3         2
0.06635 x - 0.5457 x + 2.585 x + 2.451 x - 1.282
    \end{Verbatim}

            \begin{Verbatim}[commandchars=\\\{\}]
{\color{outcolor}Out[{\color{outcolor}105}]:} [<matplotlib.lines.Line2D at 0x110be4890>]
\end{Verbatim}
        
    
    \begin{verbatim}
<matplotlib.figure.Figure at 0x10f70e850>
    \end{verbatim}

    
    \begin{center}
    \adjustimage{max size={0.9\linewidth}{0.9\paperheight}}{Instrumentacion_files/Instrumentacion_28_3.png}
    \end{center}
    { \hspace*{\fill} \\}
    
    \begin{center}
    \adjustimage{max size={0.9\linewidth}{0.9\paperheight}}{Instrumentacion_files/Instrumentacion_28_4.png}
    \end{center}
    { \hspace*{\fill} \\}
    
    \subsection{Ejemplo potenciometro}\label{ejemplo-potenciometro}

    \begin{Verbatim}[commandchars=\\\{\}]
{\color{incolor}In [{\color{incolor}82}]:} \PY{n}{Voltaje}\PY{o}{=}\PY{n}{np}\PY{o}{.}\PY{n}{array}\PY{p}{(}\PY{p}{[}\PY{l+m+mf}{3.3}\PY{p}{,} \PY{l+m+mf}{3.09}\PY{p}{,} \PY{l+m+mf}{2.8}\PY{p}{,} \PY{l+m+mf}{2.6}\PY{p}{,} \PY{l+m+mf}{2.2}\PY{p}{,} \PY{l+m+mf}{2.1}\PY{p}{,} \PY{l+m+mf}{1.9}\PY{p}{,} \PY{l+m+mf}{1.7}\PY{p}{,} \PY{l+m+mf}{1.4}\PY{p}{,} \PY{l+m+mf}{1.1}\PY{p}{,} \PY{l+m+mf}{0.8}\PY{p}{]}\PY{p}{)}
         \PY{n}{altura}\PY{o}{=}\PY{n}{np}\PY{o}{.}\PY{n}{array}\PY{p}{(}\PY{p}{[}\PY{l+m+mi}{0}\PY{p}{,} \PY{l+m+mi}{1}\PY{p}{,} \PY{l+m+mi}{6}\PY{p}{,} \PY{l+m+mi}{17}\PY{p}{,} \PY{l+m+mi}{28}\PY{p}{,} \PY{l+m+mi}{34}\PY{p}{,} \PY{l+m+mi}{44}\PY{p}{,} \PY{l+m+mi}{59}\PY{p}{,} \PY{l+m+mf}{66.5}\PY{p}{,} \PY{l+m+mi}{68}\PY{p}{,} \PY{l+m+mi}{69}\PY{p}{]}\PY{p}{)}
         \PY{n}{matplotlib}\PY{o}{.}\PY{n}{rcParams}\PY{o}{.}\PY{n}{update}\PY{p}{(}\PY{p}{\PYZob{}}\PY{l+s}{\PYZsq{}}\PY{l+s}{font.size}\PY{l+s}{\PYZsq{}}\PY{p}{:} \PY{l+m+mi}{18}\PY{p}{\PYZcb{}}\PY{p}{)}
         \PY{n}{figure}\PY{p}{(}\PY{n}{figsize}\PY{o}{=}\PY{p}{(}\PY{l+m+mi}{14}\PY{p}{,}\PY{l+m+mi}{10}\PY{p}{)}\PY{p}{,} \PY{n}{dpi}\PY{o}{=}\PY{l+m+mi}{150}\PY{p}{)}
         \PY{n}{plot}\PY{p}{(}\PY{n}{Voltaje}\PY{p}{,}\PY{n}{altura}\PY{p}{)}
\end{Verbatim}

            \begin{Verbatim}[commandchars=\\\{\}]
{\color{outcolor}Out[{\color{outcolor}82}]:} [<matplotlib.lines.Line2D at 0x10e06ecd0>]
\end{Verbatim}
        
    \begin{center}
    \adjustimage{max size={0.9\linewidth}{0.9\paperheight}}{Instrumentacion_files/Instrumentacion_30_1.png}
    \end{center}
    { \hspace*{\fill} \\}
    
    \begin{Verbatim}[commandchars=\\\{\}]
{\color{incolor}In [{\color{incolor}83}]:} \PY{n}{ze}\PY{o}{=}\PY{n}{np}\PY{o}{.}\PY{n}{polyfit}\PY{p}{(}\PY{n}{Voltaje}\PY{p}{,} \PY{n}{altura}\PY{p}{,} \PY{l+m+mi}{2}\PY{p}{)}
         \PY{k}{print}\PY{p}{(}\PY{n}{ze}\PY{p}{)}
         \PY{n}{pe} \PY{o}{=} \PY{n}{np}\PY{o}{.}\PY{n}{poly1d}\PY{p}{(}\PY{n}{ze}\PY{p}{)}
         \PY{k}{print}\PY{p}{(}\PY{n}{pe}\PY{p}{)}
         \PY{n}{altura\PYZus{}estimada}\PY{o}{=}\PY{n}{pe}\PY{p}{(}\PY{n}{Voltaje}\PY{p}{)}
         \PY{n}{figure}\PY{p}{(}\PY{n}{figsize}\PY{o}{=}\PY{p}{(}\PY{l+m+mi}{14}\PY{p}{,}\PY{l+m+mi}{10}\PY{p}{)}\PY{p}{,} \PY{n}{dpi}\PY{o}{=}\PY{l+m+mi}{150}\PY{p}{)}
         \PY{n}{plot}\PY{p}{(}\PY{n}{Voltaje}\PY{p}{,}\PY{n}{altura}\PY{p}{,}\PY{l+s}{\PYZsq{}}\PY{l+s}{r.\PYZhy{}}\PY{l+s}{\PYZsq{}}\PY{p}{)}
         \PY{n}{plot}\PY{p}{(}\PY{n}{Voltaje}\PY{p}{,}\PY{n}{altura\PYZus{}estimada}\PY{p}{,}\PY{l+s}{\PYZsq{}}\PY{l+s}{b\PYZhy{}\PYZhy{}}\PY{l+s}{\PYZsq{}}\PY{p}{)}
\end{Verbatim}

    \begin{Verbatim}[commandchars=\\\{\}]
[ -2.65540636 -22.01906455  94.88167602]
        2
-2.655 x - 22.02 x + 94.88
    \end{Verbatim}

            \begin{Verbatim}[commandchars=\\\{\}]
{\color{outcolor}Out[{\color{outcolor}83}]:} [<matplotlib.lines.Line2D at 0x10e0a3610>]
\end{Verbatim}
        
    \begin{center}
    \adjustimage{max size={0.9\linewidth}{0.9\paperheight}}{Instrumentacion_files/Instrumentacion_31_2.png}
    \end{center}
    { \hspace*{\fill} \\}
    
    \begin{Verbatim}[commandchars=\\\{\}]
{\color{incolor}In [{\color{incolor}84}]:} \PY{n}{error}\PY{o}{=}\PY{n}{altura}\PY{o}{\PYZhy{}}\PY{n}{altura\PYZus{}estimada}
         \PY{n}{figure}\PY{p}{(}\PY{n}{figsize}\PY{o}{=}\PY{p}{(}\PY{l+m+mi}{14}\PY{p}{,}\PY{l+m+mi}{10}\PY{p}{)}\PY{p}{,} \PY{n}{dpi}\PY{o}{=}\PY{l+m+mi}{150}\PY{p}{)}
         \PY{n}{plot}\PY{p}{(}\PY{n}{altura}\PY{p}{,}\PY{n}{error}\PY{p}{)}
\end{Verbatim}

            \begin{Verbatim}[commandchars=\\\{\}]
{\color{outcolor}Out[{\color{outcolor}84}]:} [<matplotlib.lines.Line2D at 0x10e88e990>]
\end{Verbatim}
        
    \begin{center}
    \adjustimage{max size={0.9\linewidth}{0.9\paperheight}}{Instrumentacion_files/Instrumentacion_32_1.png}
    \end{center}
    { \hspace*{\fill} \\}
    
    \begin{Verbatim}[commandchars=\\\{\}]
{\color{incolor}In [{\color{incolor}85}]:} \PY{n}{ze}\PY{o}{=}\PY{n}{np}\PY{o}{.}\PY{n}{polyfit}\PY{p}{(}\PY{n}{Voltaje}\PY{p}{,} \PY{n}{altura}\PY{p}{,} \PY{l+m+mi}{3}\PY{p}{)}
         \PY{k}{print}\PY{p}{(}\PY{n}{ze}\PY{p}{)}
         \PY{n}{pe} \PY{o}{=} \PY{n}{np}\PY{o}{.}\PY{n}{poly1d}\PY{p}{(}\PY{n}{ze}\PY{p}{)}
         \PY{k}{print}\PY{p}{(}\PY{n}{pe}\PY{p}{)}
         \PY{n}{altura\PYZus{}estimada}\PY{o}{=}\PY{n}{pe}\PY{p}{(}\PY{n}{Voltaje}\PY{p}{)}
         \PY{n}{figure}\PY{p}{(}\PY{n}{figsize}\PY{o}{=}\PY{p}{(}\PY{l+m+mi}{14}\PY{p}{,}\PY{l+m+mi}{10}\PY{p}{)}\PY{p}{,} \PY{n}{dpi}\PY{o}{=}\PY{l+m+mi}{150}\PY{p}{)}
         \PY{n}{plot}\PY{p}{(}\PY{n}{Voltaje}\PY{p}{,}\PY{n}{altura}\PY{p}{,}\PY{l+s}{\PYZsq{}}\PY{l+s}{r.\PYZhy{}}\PY{l+s}{\PYZsq{}}\PY{p}{)}
         \PY{n}{plot}\PY{p}{(}\PY{n}{Voltaje}\PY{p}{,}\PY{n}{altura\PYZus{}estimada}\PY{p}{,}\PY{l+s}{\PYZsq{}}\PY{l+s}{b\PYZhy{}\PYZhy{}}\PY{l+s}{\PYZsq{}}\PY{p}{)}
\end{Verbatim}

    \begin{Verbatim}[commandchars=\\\{\}]
[  13.57611741  -86.04987557  133.2932766     9.89111133]
       3         2
13.58 x - 86.05 x + 133.3 x + 9.891
    \end{Verbatim}

            \begin{Verbatim}[commandchars=\\\{\}]
{\color{outcolor}Out[{\color{outcolor}85}]:} [<matplotlib.lines.Line2D at 0x10e614b10>]
\end{Verbatim}
        
    \begin{center}
    \adjustimage{max size={0.9\linewidth}{0.9\paperheight}}{Instrumentacion_files/Instrumentacion_33_2.png}
    \end{center}
    { \hspace*{\fill} \\}
    
    \begin{Verbatim}[commandchars=\\\{\}]
{\color{incolor}In [{\color{incolor}86}]:} \PY{n}{error}\PY{o}{=}\PY{n}{altura}\PY{o}{\PYZhy{}}\PY{n}{altura\PYZus{}estimada}
         \PY{n}{figure}\PY{p}{(}\PY{n}{figsize}\PY{o}{=}\PY{p}{(}\PY{l+m+mi}{14}\PY{p}{,}\PY{l+m+mi}{10}\PY{p}{)}\PY{p}{,} \PY{n}{dpi}\PY{o}{=}\PY{l+m+mi}{150}\PY{p}{)}
         \PY{n}{plot}\PY{p}{(}\PY{n}{altura}\PY{p}{,}\PY{n}{error}\PY{p}{)}
\end{Verbatim}

            \begin{Verbatim}[commandchars=\\\{\}]
{\color{outcolor}Out[{\color{outcolor}86}]:} [<matplotlib.lines.Line2D at 0x10ed3c650>]
\end{Verbatim}
        
    \begin{center}
    \adjustimage{max size={0.9\linewidth}{0.9\paperheight}}{Instrumentacion_files/Instrumentacion_34_1.png}
    \end{center}
    { \hspace*{\fill} \\}
    
    \begin{Verbatim}[commandchars=\\\{\}]
{\color{incolor}In [{\color{incolor}87}]:} \PY{n}{ze}\PY{o}{=}\PY{n}{np}\PY{o}{.}\PY{n}{polyfit}\PY{p}{(}\PY{n}{Voltaje}\PY{p}{,} \PY{n}{altura}\PY{p}{,} \PY{l+m+mi}{10}\PY{p}{)}
         \PY{k}{print}\PY{p}{(}\PY{n}{ze}\PY{p}{)}
         \PY{n}{pe} \PY{o}{=} \PY{n}{np}\PY{o}{.}\PY{n}{poly1d}\PY{p}{(}\PY{n}{ze}\PY{p}{)}
         \PY{k}{print}\PY{p}{(}\PY{n}{pe}\PY{p}{)}
         \PY{n}{altura\PYZus{}estimada}\PY{o}{=}\PY{n}{pe}\PY{p}{(}\PY{n}{Voltaje}\PY{p}{)}
         \PY{n}{figure}\PY{p}{(}\PY{n}{figsize}\PY{o}{=}\PY{p}{(}\PY{l+m+mi}{14}\PY{p}{,}\PY{l+m+mi}{10}\PY{p}{)}\PY{p}{,} \PY{n}{dpi}\PY{o}{=}\PY{l+m+mi}{150}\PY{p}{)}
         \PY{n}{plot}\PY{p}{(}\PY{n}{Voltaje}\PY{p}{,}\PY{n}{altura}\PY{p}{,}\PY{l+s}{\PYZsq{}}\PY{l+s}{r.\PYZhy{}}\PY{l+s}{\PYZsq{}}\PY{p}{)}
         \PY{n}{plot}\PY{p}{(}\PY{n}{Voltaje}\PY{p}{,}\PY{n}{altura\PYZus{}estimada}\PY{p}{,}\PY{l+s}{\PYZsq{}}\PY{l+s}{b\PYZhy{}\PYZhy{}}\PY{l+s}{\PYZsq{}}\PY{p}{)}
\end{Verbatim}

    \begin{Verbatim}[commandchars=\\\{\}]
[ -7.87249688e+02   1.64674345e+04  -1.52523166e+05   8.22981272e+05
  -2.86192907e+06   6.69448393e+06  -1.06534967e+07   1.13722328e+07
  -7.77994794e+06   3.07437538e+06  -5.31714758e+05]
        10             9             8            7             6
-787.2 x  + 1.647e+04 x - 1.525e+05 x + 8.23e+05 x - 2.862e+06 x
              5             4             3            2
 + 6.694e+06 x - 1.065e+07 x + 1.137e+07 x - 7.78e+06 x + 3.074e+06 x - 5.317e+05
    \end{Verbatim}

            \begin{Verbatim}[commandchars=\\\{\}]
{\color{outcolor}Out[{\color{outcolor}87}]:} [<matplotlib.lines.Line2D at 0x10ed47090>]
\end{Verbatim}
        
    \begin{center}
    \adjustimage{max size={0.9\linewidth}{0.9\paperheight}}{Instrumentacion_files/Instrumentacion_35_2.png}
    \end{center}
    { \hspace*{\fill} \\}
    
    \begin{Verbatim}[commandchars=\\\{\}]
{\color{incolor}In [{\color{incolor}88}]:} \PY{n}{error}\PY{o}{=}\PY{n}{altura}\PY{o}{\PYZhy{}}\PY{n}{altura\PYZus{}estimada}
         \PY{n}{figure}\PY{p}{(}\PY{n}{figsize}\PY{o}{=}\PY{p}{(}\PY{l+m+mi}{14}\PY{p}{,}\PY{l+m+mi}{10}\PY{p}{)}\PY{p}{,} \PY{n}{dpi}\PY{o}{=}\PY{l+m+mi}{150}\PY{p}{)}
         \PY{n}{plot}\PY{p}{(}\PY{n}{altura}\PY{p}{,}\PY{n}{error}\PY{p}{)}
\end{Verbatim}

            \begin{Verbatim}[commandchars=\\\{\}]
{\color{outcolor}Out[{\color{outcolor}88}]:} [<matplotlib.lines.Line2D at 0x10f1f5490>]
\end{Verbatim}
        
    \begin{center}
    \adjustimage{max size={0.9\linewidth}{0.9\paperheight}}{Instrumentacion_files/Instrumentacion_36_1.png}
    \end{center}
    { \hspace*{\fill} \\}
    
    \begin{Verbatim}[commandchars=\\\{\}]
{\color{incolor}In [{\color{incolor}89}]:} \PY{n}{ze}\PY{o}{=}\PY{n}{np}\PY{o}{.}\PY{n}{polyfit}\PY{p}{(}\PY{n}{Voltaje}\PY{p}{,} \PY{n}{altura}\PY{p}{,} \PY{l+m+mi}{5}\PY{p}{)}
         \PY{k}{print}\PY{p}{(}\PY{n}{ze}\PY{p}{)}
         \PY{n}{pe} \PY{o}{=} \PY{n}{np}\PY{o}{.}\PY{n}{poly1d}\PY{p}{(}\PY{n}{ze}\PY{p}{)}
         \PY{k}{print}\PY{p}{(}\PY{n}{pe}\PY{p}{)}
         \PY{n}{altura\PYZus{}estimada}\PY{o}{=}\PY{n}{pe}\PY{p}{(}\PY{n}{Voltaje}\PY{p}{)}
         \PY{n}{figure}\PY{p}{(}\PY{n}{figsize}\PY{o}{=}\PY{p}{(}\PY{l+m+mi}{14}\PY{p}{,}\PY{l+m+mi}{10}\PY{p}{)}\PY{p}{,} \PY{n}{dpi}\PY{o}{=}\PY{l+m+mi}{150}\PY{p}{)}
         \PY{n}{plot}\PY{p}{(}\PY{n}{Voltaje}\PY{p}{,}\PY{n}{altura}\PY{p}{,}\PY{l+s}{\PYZsq{}}\PY{l+s}{r.\PYZhy{}}\PY{l+s}{\PYZsq{}}\PY{p}{)}
         \PY{n}{plot}\PY{p}{(}\PY{n}{Voltaje}\PY{p}{,}\PY{n}{altura\PYZus{}estimada}\PY{p}{,}\PY{l+s}{\PYZsq{}}\PY{l+s}{b\PYZhy{}\PYZhy{}}\PY{l+s}{\PYZsq{}}\PY{p}{)}
\end{Verbatim}

    \begin{Verbatim}[commandchars=\\\{\}]
[  -6.78124742   67.73220797 -243.30308654  372.80998877 -249.88809617
  128.90728071]
        5         4         3         2
-6.781 x + 67.73 x - 243.3 x + 372.8 x - 249.9 x + 128.9
    \end{Verbatim}

            \begin{Verbatim}[commandchars=\\\{\}]
{\color{outcolor}Out[{\color{outcolor}89}]:} [<matplotlib.lines.Line2D at 0x10f1ff890>]
\end{Verbatim}
        
    \begin{center}
    \adjustimage{max size={0.9\linewidth}{0.9\paperheight}}{Instrumentacion_files/Instrumentacion_37_2.png}
    \end{center}
    { \hspace*{\fill} \\}
    
    \begin{Verbatim}[commandchars=\\\{\}]
{\color{incolor}In [{\color{incolor}90}]:} \PY{n}{error}\PY{o}{=}\PY{n}{altura}\PY{o}{\PYZhy{}}\PY{n}{altura\PYZus{}estimada}
         \PY{n}{figure}\PY{p}{(}\PY{n}{figsize}\PY{o}{=}\PY{p}{(}\PY{l+m+mi}{14}\PY{p}{,}\PY{l+m+mi}{10}\PY{p}{)}\PY{p}{,} \PY{n}{dpi}\PY{o}{=}\PY{l+m+mi}{150}\PY{p}{)}
         \PY{n}{plot}\PY{p}{(}\PY{n}{altura}\PY{p}{,}\PY{n}{error}\PY{p}{)}
\end{Verbatim}

            \begin{Verbatim}[commandchars=\\\{\}]
{\color{outcolor}Out[{\color{outcolor}90}]:} [<matplotlib.lines.Line2D at 0x10f6df250>]
\end{Verbatim}
        
    \begin{center}
    \adjustimage{max size={0.9\linewidth}{0.9\paperheight}}{Instrumentacion_files/Instrumentacion_38_1.png}
    \end{center}
    { \hspace*{\fill} \\}
    
    \begin{Verbatim}[commandchars=\\\{\}]
{\color{incolor}In [{\color{incolor}91}]:} \PY{n}{ze}\PY{o}{=}\PY{n}{np}\PY{o}{.}\PY{n}{polyfit}\PY{p}{(}\PY{n}{Voltaje}\PY{p}{,} \PY{n}{altura}\PY{p}{,} \PY{l+m+mi}{10}\PY{p}{)}\PY{k}{print}\PY{p}{(}\PY{n}{ze}\PY{p}{)}
         \PY{n}{pe} \PY{o}{=} \PY{n}{np}\PY{o}{.}\PY{n}{poly1d}\PY{p}{(}\PY{n}{ze}\PY{p}{)}
         \PY{k}{print}\PY{p}{(}\PY{n}{pe}\PY{p}{)}
         \PY{n}{altura\PYZus{}estimada}\PY{o}{=}\PY{n}{pe}\PY{p}{(}\PY{n}{Voltaje}\PY{p}{)}
         \PY{n}{figure}\PY{p}{(}\PY{n}{figsize}\PY{o}{=}\PY{p}{(}\PY{l+m+mi}{14}\PY{p}{,}\PY{l+m+mi}{10}\PY{p}{)}\PY{p}{,} \PY{n}{dpi}\PY{o}{=}\PY{l+m+mi}{150}\PY{p}{)}
         \PY{n}{plot}\PY{p}{(}\PY{n}{Voltaje}\PY{p}{,}\PY{n}{altura}\PY{p}{,}\PY{l+s}{\PYZsq{}}\PY{l+s}{r.\PYZhy{}}\PY{l+s}{\PYZsq{}}\PY{p}{)}
         \PY{n}{plot}\PY{p}{(}\PY{n}{Voltaje}\PY{p}{,}\PY{n}{altura\PYZus{}estimada}\PY{p}{,}\PY{l+s}{\PYZsq{}}\PY{l+s}{b\PYZhy{}\PYZhy{}}\PY{l+s}{\PYZsq{}}\PY{p}{)}
\end{Verbatim}

    \begin{Verbatim}[commandchars=\\\{\}]
[ -7.87249688e+02   1.64674345e+04  -1.52523166e+05   8.22981272e+05
  -2.86192907e+06   6.69448393e+06  -1.06534967e+07   1.13722328e+07
  -7.77994794e+06   3.07437538e+06  -5.31714758e+05]
        10             9             8            7             6
-787.2 x  + 1.647e+04 x - 1.525e+05 x + 8.23e+05 x - 2.862e+06 x
              5             4             3            2
 + 6.694e+06 x - 1.065e+07 x + 1.137e+07 x - 7.78e+06 x + 3.074e+06 x - 5.317e+05
    \end{Verbatim}

            \begin{Verbatim}[commandchars=\\\{\}]
{\color{outcolor}Out[{\color{outcolor}91}]:} [<matplotlib.lines.Line2D at 0x10f4c5290>]
\end{Verbatim}
        
    \begin{center}
    \adjustimage{max size={0.9\linewidth}{0.9\paperheight}}{Instrumentacion_files/Instrumentacion_39_2.png}
    \end{center}
    { \hspace*{\fill} \\}
    

    % Add a bibliography block to the postdoc
    
    
    
    \end{document}
